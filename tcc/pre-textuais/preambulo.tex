%
% ********** Página de assinaturas
%

\begin{titlepage}

\begin{center}

\LARGE

\textbf{Sistema de controle para Andador inteligente}

\vfill

\Large

\textbf{Samuel Cavalcanti}

\end{center}

\vfill

% O \noindent é para eliminar a tabulação inicial que normalmente é
% colocada na primeira frase dos parágrafos
\noindent
% Descomente a opção que se aplica ao seu caso
% Note que propostas de tema de qualificação nunca têm preâmbulo.
Monografia aprovada em \today, pela banca examinadora composta
pelos seguintes membros:

% Os nomes dos membros da banca examinadora devem ser listados
% na seguinte ordem: orientador, co-orientador (caso haja),
% examinadores externos, examinadores internos. Dentro de uma mesma
% categoria, por ordem alfabética

\begin{center}

\vspace{1.5cm}\rule{0.95\linewidth}{1pt}
\parbox{0.9\linewidth}{%
Prof. Dr. XXXXX (orientador) \dotfill\ DCA/UFRN}

\vspace{1.5cm}\rule{0.95\linewidth}{1pt}
\parbox{0.9\linewidth}{%
Prof. Dr. YYYYY (co-orientador) \dotfill\ MCA/UFRN}

\vspace{1.5cm}\rule{0.95\linewidth}{1pt}
\parbox{0.9\linewidth}{%
Prof. Dr. WWWWWW \dotfill\ DEM/UFFN}

\vspace{1.5cm}\rule{0.95\linewidth}{1pt}
\parbox{0.9\linewidth}{%
Profª Drª ZZZZZZ \dotfill\ DEE/UFRN}

\end{center}

\end{titlepage}

%
% ********** Dedicatória
%

% A dedicatória não é obrigatória. Se você tem alguém ou algo que teve
% uma importância fundamental ao longo do seu curso, pode dedicar a ele(a)
% este trabalho. Geralmente não se faz dedicatória a várias pessoas: para
% isso existe a seção de agradecimentos.
% Se não quiser dedicatória, basta excluir o texto entre
% \begin{titlepage} e \end{titlepage}

% \begin{titlepage}

% \vspace*{\fill}

% \hfill
% \begin{minipage}{0.5\linewidth}
% \begin{flushright}
% \large\it
% Aos meus ......
% \end{flushright}
% \end{minipage}

% \vspace*{\fill}

% \end{titlepage}

%
% ********** Agradecimentos
%

% Os agradecimentos não são obrigatórios. Se existem pessoas que lhe
% ajudaram ao longo do seu curso, pode incluir um agradecimento.
% Se não quiser agradecimentos, basta excluir o texto após \chapter*{...}

% \chapter*{Agradecimentos}
% \thispagestyle{empty}

% \begin{trivlist}  \itemsep 2ex

% \item Ao meu orientador e ao meu co-orientador, professores XXXXXX e YYYYYY, sou grato pela orientação.

% \item Ao secretário Halidaivson Stockhouse pela ajuda no andamento das coisas burocráticas do Curso.

% \item Aos colegas ....

% \item Aos demais colegas de graduação, pelas críticas e sugestões.

% \item À minha família pelo apoio durante esta jornada.

% \item À CAPES/CNPQ, pelo apoio financeiro.

% \item À Deus.

% \end{trivlist}
