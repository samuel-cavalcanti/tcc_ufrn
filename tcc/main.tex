%%
%% Arquivo principal adaptado para:
%% - Trabalho de Conclusão de Curso de Graduação - Eng. Computação e Mecatrônica - UFRN
%%
%% NOTA: A PUBLICAÇÃO DESTE MODELO VISA APENAS ORIENTAR OS PÓS-GRADUANDOS
%% NA PREPARAÇÃO DE SEUS TEXTOS. O PPgEE DA UFRN NÃO PROVÊ ASSISTÊNCIA NO
%% USO DAS FERRAMENTAS NECESSÁRIAS AO USO DESTE MODELO (LATEX, XFIG, ETC.)
%%
%% Baseado no modelo prévio para Tese de Pós Graduação PPGEEC do Prof. Dr.Adelardo Medeiros, dezembro de 2005.
%% Revisado pelos alunos de Metodologia da Pesquisa Científica de 2016.1.


%% -------------------------------------------------------------------------
%%
%% NOTA: Modelo adaptado em Julho de 2022 para template de TCC em Eng. de Computação 
%% e Mecatrônica da UFRN/CT.
%%
%% Válber César Cavalcanti Roza (Sec. de Coordenação de Eng.Mecatrônica CTEC/UFRN),20 de Julho, 2022.
%% Revisado pelo secretariado das Coordenações dos respectivos cursos.
%%
%% -------------------------------------------------------------------------


% DEFINIÇÕES GLOBAIS

% Esta primeira linha dá informações gerais sobre o documento.
% PARA A VERSÃO FINAL:
% papel A4, letra grande (12pt), openright (capítulos só iniciam em
% página direita, se necessário incluindo uma página em branco),
% twoside (o documento vai ser impresso em frente e costa)
\documentclass[a4paper,12pt,openright,twoside]{book}
% PARA A QUALIFICAÇÃO E PARA A VERSÃO INICIAL:
% papel A4, letra grande (12pt), openany (capítulos iniciam em
% qualquer página), oneside (o documento vai ser impresso só na frente)
%\documentclass[a4paper,12pt,openany,oneside]{book}


% Use estes pacotes para poder digitar diretamente as letras com acento
% e para que a hifenização funcione corretamente
\usepackage[utf8]{inputenc}
\usepackage{ae}
% Para usar fontes standard ao invés das do LaTeX (gera melhores PDFs)
\usepackage{pslatex}
% Para a hifenização em português
\usepackage[portuges, brazil]{babel}
% Para que os primeiros parágrafos das seções também sejam indentados
\usepackage{indentfirst}
% Para poder incluir gráficos (figuras)
\usepackage{graphicx}
% Para poder fazer glossário ou lista de símbolos
% Use a segunda opção se quiser incluir na definição do símbolo a
% página e/ou a equação onde ela foi definida
\usepackage[portuguese,noprefix]{nomencl}
%\usepackage[portuguese,noprefix,refeq,refpage]{nomencl}
% Para permitir espaçamento simples, 1 1/2 e duplo
\usepackage{setspace}
% Para usar alguns comandos matemáticos avançados muito úteis
\usepackage{amsmath}
\usepackage{amsfonts}
% Para poder usar o ambiente "comment"
\usepackage{verbatim}
% Para poder ter tabelas com colunas de largura auto-ajustável
\usepackage{tabularx}
% Para executar um comando depois do fim da página corrente
\usepackage{afterpage}
% Para formatar URLs (endereços da Web)
\usepackage{url}
% Para reduzir os espaços entre os ítens (itemize, enumerate, etc.)
% Este pacote não faz parte da distribuição padrão do LaTeX.
\usepackage{lib/noitemsep}
% Para as citações bibliográficas
% Hiperreferências dentro do texto e montagem dos links do índices dos
% para os leitores de pdf (deve ser o último pacote a ser inserido).
\usepackage[breaklinks,colorlinks=false,allcolors=black,urlcolor=black]{hyperref}
% \usepackage[abbr]{lib/harvard}	% As chamadas são sempre abreviadas
\usepackage[alf]{abntex2cite}
%\harvardparenthesis{square}	% Colchetes nas chamadas
%\harvardyearparenthesis{round}	% Parêntesis nos anos das referências
%\renewcommand{\harvardand}{e}	% Substituir "&" por "e" nas referências

% PACOTES OPCIONAIS
% Para posicionar uma figura onde quiser
\usepackage{float}
% Para poder incluir arquivos Postscript com cores (do Xfig, por exemplo)
\usepackage{color}
% Para ter células em tabelas que ocupam mais de uma linha
\usepackage{multirow}
% Para poder ter tabelas longas em mais de uma página
%\usepackage{longtable}
% Para poder escrever partes do texto em "n" colunas
%\usepackage{multicol}
% Se você quiser personalizar os cabeçalhos das páginas
%\usepackage{fancyheadings}
% Para incluir algoritmos e listagens de códigos
%\usepackage{listings}
% Para incluir pseudocódigos
\usepackage[portuguese, ruled, linesnumbered]{algorithm2e}
% Capítulos com títulos em um formato "decorado"
\usepackage{lib/capitulos}

% Referência correta de ambientes flutuantes (como figuras, tabelas e algoritmos).
\usepackage[all]{hypcap}

\def \tccTitle {Sistema de controle para um Andador Robótico  inteligente}

\hypersetup{
    colorlinks=true,
    linkcolor=black,
    filecolor=magenta,      
    urlcolor=cyan,
    pdftitle={\tccTitle},
    % pdfpagemode=FullScreen,
}

% NOVOS COMANDOS

% As definições dos novos comandos estão agrupadas no arquivo "comandos.tex"
% Esta separação é opcional: se você preferir, pode por as definições
% diretamente neste arquivo
\input{lib/comandos.tex}

%
% As margens
%

% Direção horizontal

% Margem interna
% Nas páginas ímpares
\setlength{\oddsidemargin}{3.5cm}         % Margem real desejada
% Nas páginas pares
\setlength{\evensidemargin}{2.5cm}        % Margem real desejada
% Largura do texto
\setlength{\textwidth}{15cm}
% As margens laterais no LaTeX são sempre 1 polegada maiores do que as
% fixadas. Se foi fixada \setlength{\oddsidemargin}{3.5cm}, a margem
% real seria de 3.5+2.54=6.04cm. Para permitir que você não tenha que
% fazer esta conta, pode usar o número desejado nas linhas anteriores
% e a gente subtrai 1in nas próximas linhas
\addtolength{\oddsidemargin}{-1in}
\addtolength{\evensidemargin}{-1in}
% Note que a margem direita não é fixada diretamente:
% ela é obtida subtraindo-se os outros valores da largura da página.
% 3.5+15+x=21cm (largura A4) -> x = margem externa = 2.5cm

% Direção vertical

% Margem superior (entre o topo da folha e o cabeçalho), altura do
% cabeçalho e distância entre o fim do cabeçalho e o início do texto
\setlength{\topmargin}{2.0cm}             % Margem real desejada
\setlength{\headheight}{1.0cm}
\setlength{\headsep}{1.0cm}
% Altura do texto (sem cabeçalho e rodapé)
\setlength{\textheight}{22.7cm}
% Distância do fim do texto ao rodapé
\setlength{\footskip}{1.0cm}
% A margem superior no LaTeX é sempre 1 polegada maior do que a
% fixada. Se foi fixada \setlength{\topmargin}{2.0cm}, a margem
%real seria de 2.0+2.54=4.54cm. Para permitir que você não tenha que
% fazer esta conta, pode usar o número desejado na linha anterior
% e a gente subtrai 1in na próxima linha
\addtolength{\topmargin}{-1in}
% Note que a margem inferior não é fixada diretamente:
% ela é obtida subtraindo-se os outros valores, sem incluir o
% "footskip", da altura da página.
% 2.0+1.0+1.0+22.7+x=29.7cm (altura A4) -> x = margem inferior = 3cm

%
% O estilo das referências bibliográficas
%

% \bibliographystyle{bibliografia/ppgee}
\bibliographystyle{abntex2-alf}


%
% O espaçamento entre linhas
%

% As páginas iniciais são sempre em espaçamento simples
\singlespacing

% Para a criação do glossário (ou lista de símbolos)
\makenomenclature



% Inicia o texto
\begin{document}

% Paginas iniciais (sem numeração)
\pagestyle{empty}

% Página de rosto (capa interna)
%
% ********** Página de Rosto
%

% titlepage gera páginas sem numeração
\begin{titlepage}

\begin{center}

\small

\begin{tabularx}{\linewidth}{@{}l@{}C@{}r@{}}
\parbox[c]{3cm}{\includegraphics[width=\linewidth]{./figuras/UFRN}} &
\begin{center}
\textsf{\textsc{Universidade Federal do Rio Grande do Norte\\
Centro de Tecnologia\\
Graduação em Engenharia de Computação}}
\end{center} &
\parbox[c]{3cm}{\includegraphics[width=\linewidth]{./figuras/dca_logo.png}}
\end{tabularx}

\vfill
\LARGE
\textbf{Sistema de controle para Andador inteligente}
\vfill
\Large
\textbf{Samuel Cavalcanti}
\vfill

\normalsize

Orientador: Pablo Javier Alsina
% \\[2ex] Co-orientador: Prof. Dr. Zé Baiano % Opcional

\vfill

\hfill


\vfill

\large

Natal, RN, \today

\end{center}

\end{titlepage}

% Página de rosto (capa interna)
\include{pre-textuais/rosto}

% Ficha catalográfica: os dados catalográficos devem ser fornecidos
% pela BCZM.
% Só são incluídos na versão final da tese ou dissertação. Não são
% incluídos nem na proposta de tema de qualificação nem na versão
% preliminar da tese ou dissertação: nestes casos, comente a próxima linha.
%
% ********** Ficha Catalográfica
%

\newpage

\begin{center}

% Aqui não se usou \vfill porque o \vfill é construído internamente com
% o comando \vspace. Espaços verticais no início da folha com \vspace
% são ignorados. Para que isto não ocorra deve-se usar o \vspace*
% \vspace*{\fill} é como se fosse um \vfill*
\vspace*{\fill}

% Divisão de Serviços Técnicos\\[1ex]
% Catalogação da publicação na fonte.
% UFRN / Biblioteca Central Zila Mamede

Universidade Federal do Rio Grande do Norte - UFRN\\[1ex]
Sistema de Bibliotecas - SISBI\\[1ex]
Catalogação de Publicação na Fonte. UFRN - Biblioteca Central Zila Mamede
\vspace{2ex}

\begin{tabular}{|p{0.9\linewidth}|} \hline
\\ Cavalcanti, Samuel.\\
\hspace{1em} \tccTitle /
Samuel Cavalcanti - 2022 \\
\hspace{1em} 55 f.: il. \\
Monografia (graduação) - Universidade Federal do Rio Grande
do Norte, Centro de Tecnologia, Curso de Engenharia de
Computação. Natal, RN, 2022.
\\
\hspace{1em} Orientador: Pablo Javier Alsina \\
% \hspace{1em} Co-orientador: Zé Baiano \\
\\
% \hspace{1em} Monografia - Universidade Federal do Rio Grande do Norte.
% Centro de Tecnologia. Graduação em Engenharia de Computação. \\
% \\
\hspace{1em} 1. aprendizado de máquina. 2. controlador cinemático. 3. robô de acionamento diferencial. I. Alsina, Pablo Javier. II. Título.
\\
RN/UF/BCZM \hfill  CDU 004.85  \\  \hline
\end{tabular}\vspace{1cm} 
Elaborado por Fernanda de Medeiros Ferreira Aquino - CRB-15/301
\end{center}


% Assinaturas da banca, dedicatória e agradecimentos
% Só são incluídos na versão final da tese ou dissertação. Não são
% incluídos nem na proposta de tema de qualificação nem na versão
% preliminar da tese ou dissertação: nestes casos, comente a próxima linha.
%
% ********** Página de assinaturas
%

\begin{titlepage}

\begin{center}

\LARGE

\textbf{Sistema de controle para Andador inteligente}

\vfill

\Large

\textbf{Samuel Cavalcanti}

\end{center}

\vfill

% O \noindent é para eliminar a tabulação inicial que normalmente é
% colocada na primeira frase dos parágrafos
\noindent
% Descomente a opção que se aplica ao seu caso
% Note que propostas de tema de qualificação nunca têm preâmbulo.
Monografia aprovada em \today, pela banca examinadora composta
pelos seguintes membros:

% Os nomes dos membros da banca examinadora devem ser listados
% na seguinte ordem: orientador, co-orientador (caso haja),
% examinadores externos, examinadores internos. Dentro de uma mesma
% categoria, por ordem alfabética

\begin{center}

\vspace{1.5cm}\rule{0.95\linewidth}{1pt}
\parbox{0.9\linewidth}{%
Prof. Dr. XXXXX (orientador) \dotfill\ DCA/UFRN}

\vspace{1.5cm}\rule{0.95\linewidth}{1pt}
\parbox{0.9\linewidth}{%
Prof. Dr. YYYYY (co-orientador) \dotfill\ MCA/UFRN}

\vspace{1.5cm}\rule{0.95\linewidth}{1pt}
\parbox{0.9\linewidth}{%
Prof. Dr. WWWWWW \dotfill\ DEM/UFFN}

\vspace{1.5cm}\rule{0.95\linewidth}{1pt}
\parbox{0.9\linewidth}{%
Profª Drª ZZZZZZ \dotfill\ DEE/UFRN}

\end{center}

\end{titlepage}

%
% ********** Dedicatória
%

% A dedicatória não é obrigatória. Se você tem alguém ou algo que teve
% uma importância fundamental ao longo do seu curso, pode dedicar a ele(a)
% este trabalho. Geralmente não se faz dedicatória a várias pessoas: para
% isso existe a seção de agradecimentos.
% Se não quiser dedicatória, basta excluir o texto entre
% \begin{titlepage} e \end{titlepage}

% \begin{titlepage}

% \vspace*{\fill}

% \hfill
% \begin{minipage}{0.5\linewidth}
% \begin{flushright}
% \large\it
% Aos meus ......
% \end{flushright}
% \end{minipage}

% \vspace*{\fill}

% \end{titlepage}

%
% ********** Agradecimentos
%

% Os agradecimentos não são obrigatórios. Se existem pessoas que lhe
% ajudaram ao longo do seu curso, pode incluir um agradecimento.
% Se não quiser agradecimentos, basta excluir o texto após \chapter*{...}

% \chapter*{Agradecimentos}
% \thispagestyle{empty}

% \begin{trivlist}  \itemsep 2ex

% \item Ao meu orientador e ao meu co-orientador, professores XXXXXX e YYYYYY, sou grato pela orientação.

% \item Ao secretário Halidaivson Stockhouse pela ajuda no andamento das coisas burocráticas do Curso.

% \item Aos colegas ....

% \item Aos demais colegas de graduação, pelas críticas e sugestões.

% \item À minha família pelo apoio durante esta jornada.

% \item À CAPES/CNPQ, pelo apoio financeiro.

% \item À Deus.

% \end{trivlist}


%
% O espaçamento entre linhas (ATENÇÃO A ESTA PARTE!)
%
%%%%%%%%%%%%%%%%%%%%%%%%%%%%%%%%%%%%%%%%%%%%%%%%%%%%%%%%%%%%%%%%%%%%%%%%%%%%
% PARA A VERSÃO FINAL:
% Deve ser usado espaçamento simples nas páginas de texto
\singlespacing
% PARA A QUALIFICAÇÃO E PARA A VERSÃO INICIAL:
% Deve ser usado espaçamento 1 1/2 nas páginas de texto
%\doublespacing
%%%%%%%%%%%%%%%%%%%%%%%%%%%%%%%%%%%%%%%%%%%%%%%%%%%%%%%%%%%%%%%%%%%%%%%%%%%%

% Resumo/Abstract
%
% ********** Resumo
%

% Usa-se \chapter*, e não \chapter, porque este "capítulo" não deve
% ser numerado.
% Na maioria das vezes, ao invés dos comandos LaTeX \chapter e \chapter*,
% deve-se usar as nossas versões definidas no arquivo comandos.tex,
% \mychapter e \mychapterast. Isto porque os comandos LaTeX têm um erro
% que faz com que eles sempre coloquem o número da página no rodapé na
% primeira página do capítulo, mesmo que o estilo que estejamos usando
% para numeração seja outro.
\mychapterast{Resumo}
O principal objetivo deste trabalho é criar um  sistema de
controle para o andador robótico inteligente que seja capaz de
posicionar o robô em uma posição em um ambiente livre de obstáculos,
também  este trabalho visa estudar e avaliar algoritmos de aprendizado
de máquina na elaboração do controlador. Primeiro foi feito um levantamento
literário sobre a utilização aprendizado de máquina para criação de
controladores cinemáticos em robôs moveis de acionamento diferencial,
segundo foi construído uma versão do robô no simulador Coppeliasim,
terceiro foi projetado um controlador cinemático, onde dois modelos
cinemáticos foram obtidos através da modelagem dos parâmetros de uma
rede neural artificial utilizando as equações que limitam o movimento
cinemático do robô, um modelo faz uma regressão não linear dos
parâmetros das equações, já outro modelo realiza uma regressão linear.
Os dados utilizados para o treinamento da rede são obtidos
através da criação de  algoritmos de coleta de dados e pré-processamento.
Ao todo foram coletados dez mil
amostras para o treinamento. Já no controlador utiliza uma abordagem
clássica que possui  dois controladores
Proporcionais, integrais, derivativos (P.I.D),
onde ambos os parâmetros dos P.I.Ds foram obtidos empiricamente.
Para avaliar os modelos cinemáticos foi calculado o modelo analítico,
também foi coletado um conjunto de
dados de testes que foi utilizado para o cálculo do erro quadrático
médio para todos os modelos cinemáticos.
Para avaliar todo o sistema foi posicionado um alvo e observado o robô no 
seu trabalho em chegar até alvo mensurando a distância do robô até o
alvo e o seu angulo ao longo do tempo, o alvo foi posicionado em 4 lugares
diferentes. Os resultados do erro quadrático médio e os
gráficos de distância e angulo sobre o tempo 
mostraram que os dois modelos cinemáticos são equivalentes ao modelo
analítico. Comparando os parâmetros do modelo que realiza a
regressão não linear com o modelo analítico,
percebe-se os parâmetros como raio da roda e a distância entre as rodas,
se aproximaram da solução analítica, já para os ângulos das rodas foram
encontrados outros valores que quando somados se aproximam da solução
analítica. Foi concluído durante o levantamento literário que
utilizar a abordagem clássica é a melhor opção do que soluções com
aprendizado de máquina devido ao uso de memória e,
processamento com ambas soluções apresentando resultados equiparáveis.
Foi concluído que projetar o controlador cinemático encontrando o modelo
de forma analítica é mais simples do que usando algoritmos de aprendizado
de máquina, no entanto aplicar a modelagem  de sistemas para criar um
modelo de rede neural artificial resultou em um desempenho equivalente
a solução analítica.




\vspace{1.5ex}

{\bf Palavras-chave}: aprendizado de máquina, robô de acionamento diferencial,
controlador cinemático.

%
% ********** Abstract
%
\mychapterast{Abstract}

Lorem ipsum dolor sit amet, consectetur adipiscing elit. Donec vehicula vitae lectus ut pretium. Vestibulum tristique leo eu purus vehicula ullamcorper. Nulla ut ultricies massa. Suspendisse eu neque pharetra, faucibus erat ac, pretium augue. Vivamus id euismod leo. Cras eget neque pellentesque, fringilla dolor eu, pretium libero. Mauris sed justo feugiat, varius ligula sed, posuere metus. Fusce lacus mi, molestie a rutrum id, scelerisque ut lacus. In hac habitasse platea dictumst. In vitae elit faucibus, molestie orci efficitur, consectetur neque. Ut placerat, augue eu pellentesque euismod, dui enim euismod elit, quis sollicitudin lectus lorem gravida mi. Donec ut leo pretium, finibus arcu in, tincidunt sem. Phasellus diam ante, pulvinar vel neque non, sagittis aliquam nibh. Praesent id condimentum nunc, quis interdum metus. Curabitur eget diam vitae enim consequat mollis quis dictum turpis.

\vspace{1.5ex}

{\bf Keywords}: Document Processing, \LaTeX, Thesis Preparation,
Technical Reports.


% Paginas introdutórias (com numeração romana)
\frontmatter

% Lista de conteúdo (sumário, gerado automaticamente)
% Aqui, e em todos os itens antes da introdução, o comando \phantomsection é utilizado.
% O seu uso é neecssário para auxiliar o pacote "hyperref" a fazer a referência correta
% dos links do sumário, das listas (de tabelas, figuras, algoritmos) com as páginas
% respectivas.
% Caso seja tirado, o "hyperref" irá apontar o link do sumário para o abstract, o link
% do sumário para a lista de figuras, o link das listas de figuras para a lista de tabelas,
% e assim sucessivamente.
\phantomsection
\addcontentsline{toc}{chapter}{Sumário}
\tableofcontents

% Lista de figuras (gerada automaticamente)
\cleardoublepage
\phantomsection
\addcontentsline{toc}{chapter}{Lista de Figuras}
\listoffigures

% Lista de tabelas (gerada automaticamente)
\cleardoublepage
\phantomsection
\addcontentsline{toc}{chapter}{Lista de Tabelas}
\listoftables

% Páginas do texto principal (com cabeçalho)
\mainmatter
\pagestyle{headings}

% Para facilitar a organização, foi criado um diretório para cada
% capítulo do documento, pois assim os arquivos das figuras ficam
% classificados por capítulos

% Cap. 1 - Introdução
%%
\mychapter{Introdução}
\label{Cap:Introducao}

\section{Motivação}

Um Andador Robótico Inteligente foi projetado e implementado
por Alberto Tavares de Oliveira \cite{oliveira2022projeto},
com objetivo de ajudar pessoas com mobilidade reduzida. Alberto aponta
diversos fatores que podem provocar uma redução da mobilidade, além do 
envelhecimento da população mundial. Apesar da construção,
o robô não possui um sistema de controle. Segundo \cite{siegwart2011introduction}
um sistema de controle de um robô móvel é um sistema que recebe
como entrada uma posição e orientação desejada e movimenta o robô
para esta posição e orientação. Um sistema de controle cinemático possui
três componentes: um controlador, um modelo cinemático do robô e sensor de
posição e orientação. O controlador é o componente responsável
por receber uma posição e orientação desejada e enviar sinais de velocidade linear 
e angular para o modelo cinemático, sabendo que as velocidades levaram o 
robô até a posição e orientação desejada. O modelo cinemático é uma função
capaz de relacionar a velocidade linear e angular do robô para as velocidades
das rodas. Aplicando as velocidades das rodas no robô, o movimento do
robô é percebido pelo sensores de posição e orientação. Dependendo do tipo de controlador
a informação dos sensores pode ou não ser utilizada pelo controlador. Este trabalho visa
utilizar aprendizado de máquina para a construção de um sistema de controle cinemático.
Aprendizado de máquina é todo sistema de computador que é capaz de
automaticamente melhorar o seu desempenho em resolver um problema através
da experiência \cite{mitchell1990machine}.
O desejo de utilizar sistemas de aprendizado de máquina é devido ao seu sucesso
em
resolver jogos como o sistema alphazero \cite{silver2017mastering}, muzero
\cite{schrittwieser2020mastering} e outros sistemas de aprendizado de máquina
que resolvem a cinemática de braços robóticos como \cite{cavalcanti2017self}
ou sistemas ainda mais complexos que resolvem a dinâmica de um robô humanoide
\cite{phaniteja2017deep}. O estudo da cinemática é o estudo do comportamento
básico de como um sistema mecânico se comporta. A dinâmica é o entendimento
do comportamento perante as forças aplicadas em um sistema mecânico. Em contexto
de robôs moveis o entendimento da cinemática é fundamental para criação de um
sistema de controle \cite{siegwart2011introduction}. Este trabalho tem como
motivação o sucesso de aplicações de aprendizado de máquina em resolver problemas
complexos e a demanda por um sistema de controle para um andador
inteligente que facilite a locomoção de pessoas com mobilidade reduzida. 

\section{Objetivos}

O principal objetivo deste trabalho é criar um  sistema de
controle cinemático para o andador inteligente. O sistema deverá
ser pensado para utilizar pouca e memória e processamento, de modo
que ele futuramente possa ser embarcado em um minicomputador.
Também este trabalho visa utilizar e avaliar o desempenho de algoritmos
de aprendizado de máquina para resolver a cinemática do robô. Tendo em
vista possíveis dificuldades encontradas em avaliar e utilizar o robô real,
um dos objetivos deste trabalho é criar uma versão simulada do robô.
Todos os testes e avaliações do sistema de controle cinemático deveram
ser feitas no simulador.

\section{organização do trabalho}

Este trabalho está organizado da seguinte forma, no capítulo \ref{Cap:Teoria}
está a revisão teoria necessária para o entendimento desse trabalho. O capítulo
\ref{Cap:TrabalhosRelacionados} é revisão bibliográfica que inspirou este
trabalho. No capítulo \ref{Cap:Desenvolvimento} é dito em detalhes o que foi
utilizado, as minhas contribuições e
como foi desenvolvido o sistema de controle cinemático. No capítulo
\ref{Cap:ExperimentosResultados}
está os experimentos feitos com o sistema de controle cinemático
e como os modelos cinemáticos gerados a partir do sistema de aprendizado de
máquina.
No capítulo \ref{Cap:ExperimentosResultados} também está uma discussão feita
a partir dos resultados dos experimentos. Por fim  no capítulo \ref{Cap:Conclusao} está a conclusão deste
trabalho.

% Cap. 2 - Teoria (Fundamentação Teórica)
%%
%% Capítulo 2: Regras gerais de estilo
%%

\mychapter{Fundamentação Teórica}
\label{Cap:Teoria}
Este trabalho envolve a cinemática de um robô móvel,
algoritmos de aprendizado de máquina e
a implementação de um controlador estabilizante
que utiliza as leis de controle. Portanto esta seção
visa explicar as partes fundamentais destas areas do conhecimento,
que foram utilizada neste trabalho. Primeiro abordaremos o problema
da cinemática focando em robôs moveis de acionamento diferencial,
segundo abordaremos o aprendizado de máquina supervisionado
relacionando o tema com a cinemática, terceiro explicaremos o
funcionamento do controlador cinemático estabilizante criado
por Federico Vieira \cite{vieira2006controle}.

\section{O problema da cinemática de um robô móvel}
Um robô de acionamento diferencial, onde as velocidades
angulares das $n$ rodas são: $\phi_0,\phi_1,\phi_2,...,\phi_n$.
Uma cinemática direta $f$ de um robô móvel é a uma função que mapeia as
velocidades
angulares das rodas para a velocidade linear $v$ e velocidade angular $\omega$
do robô, ou seja, $f(\phi_0,\phi_1,\phi_2,...,\phi_n) \rightarrow v,\omega$. Já
a cinemática inversa $g$, mapeia a velocidade angular e linear do robô para
um
conjunto de velocidades angulares das rodas,
$g(v,\omega) \rightarrow  \phi_0,\phi_1,\phi_2,...,\phi_n$.
Cinemática é portanto um conjunto de regras que relaciona a velocidade
linear $v$ e angular $\omega$
com as velocidades das rodas $\phi$.

\begin{figure}[H]
    \label{fig:robo:movel:acionamento:diferencial}
    \centering
    \includegraphics[scale=0.9]{figuras/robo.png}%
    \caption{Robô móvel com acionamento diferencial}
\end{figure}

A cinemática de um robô móvel está fortemente relacionada a modelagem
das rodas do robô. Para um robô com duas rodas de acionamento diferencial
como mostrado na figura \ref{fig:robo:movel:acionamento:diferencial}
as equações que regem o movimento dele são:

\begin{equation}\label{eq:cinmeatica:1}
    \begin{bmatrix}
        \sin(\alpha_{1} + \beta_{1}) &  -\cos(\alpha_{1} + \beta_{1}) &  -l_1\cos(\beta_{1})\\
        \sin(\alpha_{2} + \beta_{2}) &  -\cos(\alpha_{2} + \beta_{2}) &  -l_2\cos(\beta_{2})\\
    \end{bmatrix}
    \begin{bmatrix}
        v_x \\
        v_y \\
        \omega\\
    \end{bmatrix}
    =
    \begin{bmatrix}
        r_0\phi_0 \\
        r_1\phi_1 \\
    \end{bmatrix}
\end{equation}


\begin{equation}\label{eq:cinmeatica:2}
    \begin{bmatrix}
        \cos(\alpha_{1} + \beta_{1}) &  \sin(\alpha_{1} + \beta_{1}) &  l_n\sin(\beta_{1}) \\
        \cos(\alpha_{2} + \beta_{2}) &  \sin(\alpha_{2} + \beta_{2})  &  l_n\sin(\beta_{2})\\
    \end{bmatrix}
    \begin{bmatrix}
        v_x \\
        v_y \\
        \omega\\
    \end{bmatrix}
    =
    \begin{bmatrix}
        0 \\
        0 \\
    \end{bmatrix}
\end{equation}

\begin{figure}[H]
    \label{fig:parametros:roda:fixa}
    \centering
    \includegraphics[width=10cm]{figuras/fixed_wheel_params.png}
    \caption{Parâmetros de uma roda fixa}
\end{figure}
onde $\alpha$, $\beta$, $l$,$r$ são propriedades das rodas do robô,
$v_x$,$v_y$ são as velocidade lineares ao longo dos eixos x,y.
É importante salientar que os parâmetros das
rodas são extraídos seguindo o referencial do robô $X_r$,$Y_r$ assim como
as velocidades.
\begin{figure}[H]
    \centering
    % \includegraphics[scale=0.9]{figuras/robo.png}
    \includegraphics[scale=0.28]{figuras/robo_coordenadas.png}
   
    \[
    \begin{bmatrix}
        v_x \\
        v_y \\
        \omega
    \end{bmatrix}
    =
    \begin{bmatrix}
        cos(\theta)  & sin(\theta) & 0 \\
        -sin(\theta) & cos(\theta) & 0 \\
        0 & 0 & 1 \\
    \end{bmatrix}
    \begin{bmatrix}
        v_{x_I} \\
        v_{y_I} \\
        \omega
    \end{bmatrix}
\]
\caption{Mudança de referencial}
\label{fig:mudanca:referencial}
\end{figure}



Caso a velocidade linear do robô for medida em outro
referencial, é necessário fazer uma rotação do referencial
medido para a orientação $X_r$,$Y_r$. Na figura \ref{fig:mudanca:referencial}
podemos observar um exemplo de mudança de referencial,
onde $v_{x_I}, v_{y_I}$, são as velocidades no referencial $X_I,Y_I$.
Tanto a figura \ref{fig:mudanca:referencial} e \ref{fig:parametros:roda:fixa}
foram retirados do livro \cite{siegwart2011introduction}

\begin{figure}[H]
    \centering
    \includegraphics[scale=0.4]{figuras/smart_walker.png}
    \caption{Andador robótico inteligente}
    \label{fig:andador:robotico:inteligente}
\end{figure}


No nosso caso, o robô a ser modelado possui quatro rodas. Duas delas são
rodas de acionamento diferencial e as outras duas rodas são rodas castores.
Uma imagem do robô pode ser vista na figura \ref{fig:andador:robotico:inteligente}.
Apesar das rodas castores, contribuírem para o movimento, elas não restringem
o movimento, portanto podemos entender o modelo cinemático do nosso robô
como sendo uma transformação linear $W \times X = Y$, onde $X$ é o vetor
de velocidades, e $W$ a matriz de valores constantes obtidas após resolver
as equações: \ref{eq:cinmeatica:1} e \ref{eq:cinmeatica:2}.

\begin{figure}[H]
    \[
    \begin{bmatrix}
        W_{11} &  W_{12} & W_{13} \\
        W_{21} &  W_{22} & W_{23} \\
    \end{bmatrix}
    \begin{bmatrix}
        v_x \\
        v_x \\
        \omega \\
    \end{bmatrix}
    =
    \begin{bmatrix}
        \phi_{\text{left}} \\
        \phi_{\text{right}} \\
    \end{bmatrix}
\]
    \caption{Cinemática inversa}
\end{figure}

Um fato muito importante sobre a cinemática do robô de acionamento diferencial
é que resolvendo a cinemática analaticamente vamos perceber que
$\alpha + \beta = 90^{\circ}$ e $\sin(\beta) = 0$, portanto a equação:

\begin{equation}
    v_x\cos(\alpha + \beta)  + v_y\sin(\alpha + \beta) +  \omega l_n\sin(\beta) = 0
\end{equation}

irá se traduzir em:
\begin{equation}
    v_y = 0
\end{equation}
ou seja, segundo a solução analítica não é possível que
o robô translade no eixo y.

\section{Aprendizado de máquina}
Usualmente quando programamos computadores para
resolver determinada tarefa, nós codificamos as regras e
executamos o programa com os dados de entrada e obtemos a resposta.
Esta é a abordagem clássica de se resolver um problema. Uma outra abordagem
é criar sistemas de aprendizado máquina, onde nós possuímos os
dados de entrada, as respostas e queremos que a máquina retorne
para nós quais são as regras que transformam os dados nas respostas
desejadas \cite{chollet2021deep}.

\begin{figure}[H]
    \centering
    \includegraphics[scale=0.7]{figuras/machine_learning_diagram.pdf}
    \caption{Aprendizado de máquina e Programação clássica}
\end{figure}

Existem três tipos de aprendizado de máquina: aprendizado por reforço,
aprendizado supervisionado e aprendizado não supervisionado.
Aprendizado por reforço é o aprendizado de como mapear situações
para ações de modo que maximize um sinal de recompensa
\cite{sutton2018reinforcement}. Aprendizado supervisionado é o
tipo de aprendizado onde a máquina busca extrair padrões nos conjuntos
de dados de entrada  e do conjunto de dados das respostas de modo que
transforme um dado de entrada na resposta desejada. Aprendizado não
supervisionado também busca encontrar padrões entre os conjuntos de
dados de entrada e resposta, mas sem ter uma resposta correta
\cite{trask2019grokking}. Neste trabalho foi utilizado algoritmos de
aprendizado de máquina supervisionado com objetivo de extrair um modelo de
cinemática inversa do robô andador inteligente.

\begin{figure}[H]
    \centering
    \includegraphics[scale=0.7]{figuras/aprendizado_cinemática_inversa.pdf}
    \caption{Aprendizado de máquina no calculo da cinemática inversa}
\end{figure}

Os algoritmos de aprendizado supervisionado utilizados neste trabalho
foram o Backpropagation e uma variação do algoritmo decida do gradiente, chamada
RMSprop.
Dado uma função $L(M)$, onde $L$ é uma função contínua e derivável,
a decida do gradiente é uma busca que visa encontrar
a melhor matriz $M$ que minimiza uma função $L(M)$, basicamente a
decida do gradiente parte do suposto que existe a sequência
$M_0$, $M_1$, $M_2$, ..., $M_i$, $M_{i+1}$ até $M_k$, onde quando
chegar até $M_k$, o ponto: $(L(M_k),M_k)$ será um ponto de mínimo da função $L$ e a transição
entre o $M_{i}$ até $M_{i+1}$, segue a seguinte regra:

\begin{equation}
    M_{i+1} = M_i -\alpha \nabla L(M_i)
\end{equation}

onde $\nabla L(W_i)$ é o gradiente da função $L$ em relação a $M$ aplicado
a $M_i$ e $\alpha$ é um número que varia de 0 até 1,
também conhecido com taxa de aprendizado. Uma variação desse algoritmo
bastante utilizado é chamado RMSprop, ele se inspira na ideia de momentum
da física  onde  é adicionado uma constante $m$ análoga a massa e uma grandeza
vetorial $v_{\text{vel}}$ análoga a velocidade, criando uma nova equação de decida:

\begin{figure}[H]
    \begin{align*}
        v_{\text{vel}_t} = \alpha v_{\text{vel}_{t-1}} + (1 - \alpha)\nabla L(M_i)^2 \\
        b_t = mb_{t-1} + \frac{\nabla L(M_i)}{\sqrt{v_{\text{vel}_t} + \epsilon}} \\
        M_{i+1} = M_i - b_t
    \end{align*}
    \caption{Equações do RMSprop}
\end{figure}

As variáveis $v_{\text{vel}_t}$ e  $b_t$, são inicializadas como zero, pode acontecer
que  $\sqrt{v_{\text{vel}_t}}$ seja zero ou bem próximo disso então para não ter uma equação
divida por zero é adicionado uma variável $\epsilon =10^{-8}$, deixando o algoritmo numericamente
mais estável. Em contexto de aprendizado supervisionado a função $L$ é chamada de função erro, que 
através do RMSprop os parâmetros dos modelo são guiados no sentido de minimizar o seu valor.
No entanto não se possui diretamente o gradiente de $L$ em relação aos parâmetros, para isso
é utilizado o  Backpropagation que encontra os gradientes a partir
de pontos $X$ e $Y_{\text{true}}$ coletados. O conjunto de pontos $X$ são
os dados de entrada
e o conjunto de pontos  $Y_{\text{true}}$ são as respostas desejadas.
Partindo de  uma definição de um modelo como por exemplo uma
transformação geométrica $Y_{\text{pred}}= A \times X + B$, e uma definição de uma função
erro como: $L=(Y_{\text{true}}- Y_{\text{pred}})^2$ podemos utilizar o algoritmo
Backpropagation para encontrar os gradientes da função $L$ em relação a $A$
e  $B$. Basicamente o Backpropagation executa automaticamente a regra da cadeia
para encontrar os gradientes e com os gradientes podemos executar o RMSprop,
ou seja, $A_{i+1} = A_i -b_{t_a}$ e $B_{i+1} = B_i -b_{t_b}$. Onde $b_{t_a}$ e $b_{t_b}$
são os valores de $b_{t}$ do RMSprop das respectivas matrizes $A$ e $B$.

\begin{figure}[H]
    \begin{align*}
        \frac{dL}{dA } = \frac{dL}{dY_{\text{pred}} } \frac{dY_{\text{pred}}}{dA}  \\
        \frac{dL}{dB } = \frac{dL}{dY_{\text{pred}} } \frac{dY_{\text{pred}}}{dB} 
    \end{align*}
    \caption{Regra da cadeia aplicada ao modelo: $Y_{\text{pred}}= A \times X + B$ }
\end{figure}

\section{Controlador estabilizante Vieira}
Controladores cinemáticos ou controladores de movimento, podem ser 
divididos em dois tipos: seguidores de trajetória e estabilizante.
Controladores seguidores trajetória produzem um perfil
de caminho sobre o tempo, a qual o controlador envia sinais de
velocidade linear e angular para o robô de modo que ele siga a
trajetória idealizada. 
A trajetória é feita a partir da posição,orientação atual do robô e uma posição,orientação desejada.
Controladores estabilizantes eles recebem como entrada um estado atual do robô e o estado desejado
e buscam enviar sinais de velocidade linear e angular com o objetivo
de minimizar o erro entre o estado atual e desejado. Neste trabalho
foi utilizado um controlador estabilizante criado por Frederico
\cite{vieira2006controle}, o controlador recebe como entrada a posição
$p_c$  no plano x, y, uma orientação $\theta_c$ atual do robô e uma
posição desejada $p_d$ e enviar sinais de velocidade linear $v$ e
angular $\omega$ com o objetivo de estabilizar o robô no ponto desejado.

\begin{figure}[H]
    \centering
    \includegraphics[scale=0.8]{figuras/controlador_viera.pdf}
    \caption{Bloco controlador Vieira}
\end{figure}

Internamente o controlador utiliza  dois controladores proporcionais
integrais derivativos (P.I.D). Um controlador é responsável pela
velocidade linear $v$
e outro controlador é responsável pela velocidade angular $\omega$. 
Um controlador P.I.D é definido pela a seguinte função de
transferência $C(s)$ no domínio continuo $s$
\begin{equation}
    C(s) = K_p + \frac{K_i}{s} + K_ds
\end{equation}
onde $K_p$,$K_i$,$K_d$ são constantes que podem ser adquiridas ou
através da experimentação ou uma analise matemática \cite{ogata2010modern}. Para gerar sinais
velocidades $v$ e $\omega$, o controlador Vieira funciona da seguinte
maneira. Primeiro é calculado o vetor posição $p_{\text{diff}}$:
\[
    p_{\text{diff}} = p_d - p_c 
\]
segundo são calculadas suas coordenadas polares $l$, $\alpha$.

\[
    l = \sqrt{p_{\text{diff}_x}^2 + p_{\text{diff}_y}^2}
\]
onde $p_{\text{diff}_x}$,$p_{\text{diff}_y}$ são as coordenadas x,y
do vetor $p_{\text{diff}}$.  Terceiro é calculado o angulo $\gamma$:
\[
    \alpha =  \arctan(\frac{ p_{\text{diff}_y}}{p_{\text{diff}_x}}) 
\]

\[
    \gamma =  \alpha - \theta
\]
Por fim o controlador P.I.D de velocidade
linear, busca enviar um sinal $v$ que faça tender o valor $l \cos(\gamma)$
a zero e o controlador de velocidade angular busca enviar um sinal $\omega$
para  tender o valor de $\gamma$ a zero. Perceba que quando o valor de $\gamma$
tende a zero, a velocidade linear vai tender a reduzir apenas
distância $l$. Como o modelo cinemático do robô espera receber como entrada
um vetor de velocidade linear em coordenadas cartesianas então é preciso
transformar de volta a velocidade $v$ 

\[
    \begin{bmatrix}
        v_x \\
        v_y \\
    \end{bmatrix}
    =
    \begin{bmatrix}
        v\cos(\theta) \\
        v\sin(\theta) \\
    \end{bmatrix}
\]
onde, $v_x$, $v_y$ são as velocidade lineares em coordenadas cartesianas.

% Cap. 3 - Trabalhos Relacionados
%%
%% Capítulo 2: Expressões matemáticas
%%

\mychapter{Trabalhos relacionados}
\label{Cap:TrabalhosRelacionados}


\section{MuZero}

MuZero é um sistema de máquina que utiliza algoritmos de
aprendizado por reforço e aprendizado supervisionado para
resolver jogos de atari, xadrez e Shogi também conhecido como
xadrez japonês, o MuZero é um sistema com três componentes:
Representação, dinâmica e predição, Representação é uma uma
função $h_{\theta}$  que recebe $t$ observações 
e cria uma Representação interna $s$, $s = h_{\theta}(o_1,...,o_t)$,
onde $o$ é uma observação, uma observação pode ser entendia como sendo
um vetor que contem todos os valores dos sensores em um instante $t$.
A predição é uma função $f_{\theta}$ que recebe uma representação $s$
e retorna uma vetor de valores $v[k]$ que dizem o quão boa seria tomar
uma decisão discreta $k$,  $v[k] =  f_{\theta}(s)$ ,
para cada ação discreta $k$, uma ação $a$,poderia ser o índice do maior
valor de $v^{k}$ ou utilizar a busca em árvore do Monte Carlo através do
modelo de dinâmica, o modelo de dinâmica é uma outra função
$g_{\theta}$, que busca prever dado uma ação $a^k$ e um estado $s^{k}$,
encontra o estado $s^{k+1}$ e sua recompensa $r^{k+1}$ associado ao estado,
ou seja, $s^{k+1},r^{k+1}=  g_{\theta}(s^{k},a^k)$.


\begin{figure}[H]
    \centering
    \includegraphics[scale=0.4]{figuras/muzero_dig.pdf}
    \caption{diagrama dos modelos do Muzero}
\end{figure}

A importância do MuZero para este trabalho não se trata do modelo em si,
mas sim a forma como resolveu o problema, dividindo o sistema em componentes
de forma muito semelhante  em teoria de controle, onde
o modelo de predição  $f_{\theta}$ pode ser entendido algo
análogo ao controlador,o  modelo dinâmico $g_{\theta}$ como a planta,
e $h_{\theta}$ sendo os sensores, sendo $h_{\theta}$ o modelo
necessário  para que o sistema seja genérico o suficiente para jogar
diferentes jogos. É importante destacar a diferença entre o controlador P.I.D
toma decisões a partir de instantes $t_n,t_{n-1},...t_{0}$, já o $f_{\theta}$
com a buscar Monte Carlo toma decisões em $k$ instantes futuros utilizando
a função $g_{\theta}$ análoga a planta.

\begin{figure}[H]
    \centering
    \includegraphics[scale=0.6]{figuras/sistema_classico_controle.pdf}
    \caption{diagrama clássico de controle}
\end{figure}

\section{Controlador de posição e aprendizado por reforço }

Utilização de aprendizado por reforço para criação de controladores
de retroalimentação, foi observado em \cite{farias2020position}
a qual mostrou que o algoritmo \textit{Q-learning} é capaz de aprender
as regras de um controlador com estrições de velocidades linear e angular
para  posicionar o robô em determinada posição, no entanto em seus
experimentos foram necessários 5 milhões de iterações para se ter um modelo
inferior a um controlador que utiliza as leis clássicas de controle.

\subsection{Q-learning}
\textit{Q-learning} é um algoritmo de aprendizado por reforço que
busca aprender uma função ação-valor $Q$ que diretamente aproxima
a função ação-valor ótima $Q^*$ independe do algoritmo utilizado
para selecionar uma ação. \textit{Q-learning} é definido:

\begin{equation} 
    Q(S_t,A_t) \leftarrow Q(S_t,A_t) + \alpha[R + \gamma  \max_aQ(S_{t +1},a) - Q(S_t,A_t)]
\end{equation}
onde $S_t$ e $A_t$ é respectivamente estado,ação,  $ \max_a Q(S_{t +1},a)$ é o valor da
ação de maior valor no estado $S_{t+1}$ e as constantes $\alpha$, $\gamma$ são taxas
de aprendizado que variam de 0 até 1. É muito comum representar a função $Q$
sendo uma tabela, onde o estado $S$ e ação $A$ são indices, e em contexto de
controladores de retroalimentação para robôs moveis, o estado pode ser um
valor de distância até o alvo discretizado e ação pode ser movimentos simples
como girar para esquerda,direita, ir para frente e rê. Por fim o valor é
número continuo ou discreto derivado de uma função recompensa que poder
a a variação de distância até o alvo observada após o movimento. O algoritmo
Q-learning é apresentado em Algoritmo  \ref{Q-learning:}


\begin{algorithm}[H]
    \label{Q-learning:}
    Parâmetros dos algoritmo: importância de um paço $\alpha \in (0,1]$
    taxa de aprendizado $\gamma \in (0,1] $

    
    \Entrada{$N_e$:número de epsodios; $N_p$ = número de paços}
    %% \SetLine
    
    inicialize $Q(s,a)$ arbitrariamente exceto $Q(\text{terminal},\cdot ) = 0$

    \Para {$e \leftarrow 0$ \Ate $N_e$} {
        inicialize o $S$ com o estado inicial $S_0$

        
        \Para {$j \leftarrow 0$ \Ate $N_p$} {
            Escolha a ação $A$ do estado $S$ usando alguma regra derivada de $Q$,
            por exemplo a $\epsilon$-greedy

            Realize a ação $A$ e observe a recompensa $R$ e o novo estado $S'$
            
            $Q(S,A) \leftarrow (S,A) + \alpha[R + \gamma \max_aQ(S',a) - Q(S,A)]$
            
            $S \leftarrow S'$


            \Se {S é terminal}{
                
              $j \leftarrow N_p$  \textbf{ fim do loop de paços}
            }
        }
        
    }
    \Retorna $Q(s,a)$
    \caption{Algoritmo Q-learning}
    
\end{algorithm}

Em \cite{quiroga2022position}
foram utilizado algoritmos variantes do \textit{Q-learning}, como \textit{Deep Q-learning Network} e \textit{Deep Deterministic Policy Gradient},
para encontrar as regras de um controlador que posicione o robô em determinada
posição e evite de obstáculos usando sensores ultrassônicos, apesar do sucesso
os algoritmos não enviam os sinais de velocidades angulares
diretamente para o modelo cinemático do robô, os sinais
eram a entrada do algoritmo de Braitenberg, um algoritmo que converte
sinais de distância de obstáculos em velocidades lineares e angulares
que evitam a colisão com obstáculos, além do treinamento permitir que o robô
colida com um obstáculo foi observado que o tempo de treinamento desses modelos
foram de 8,6 e 11,4 horas de treinamento para \textit{Deep Q-learning Network}
e \textit{Deep Deterministic Policy Gradient}, respectivamente, segundo o
artigo o melhor o controlador foi o \textit{Deep Deterministic Policy Gradient}
que conseguiu chegar na posição desejada 3 segundos mais rápido do que o
melhor controlador que utiliza as leis clássicas de controle isso em um ambiente sem
obstáculos e em um ambiente com obstáculos a diferença aumentou para
17,3 segundos, no entanto a distância do robô até os obstáculos ficou
inferior a 20 centímetros, ou seja, o modelo de aprendizado encurtou o
caminho ficando mais próximo de bater no obstáculo, é importante dizer que
em ambos os trabalhos utilizaram um robô de acionamento diferencial.

\subsection{Deep Q-learning e Deep Deterministic Policy Gradient }
\textit{Deep Q-learning Network} é o algoritmo \textit{Q-learning}
aplicado a redes neurais de modo que a função gerada pelo treinamento
dessa rede é uma aproximação da função valor ótima $Q^*$, uma das vantagens
de utilizar redes neurais é fato de poder utilizar estados continuos sem a
necessidade de discretização, no entanto o espaço de ação ainda é necessário
ser discreto. Muitas das vezes esse algoritmo é utilizado
tendo como estado $S$ sendo uma imagem que normalmente é preprocessada
antes de enviada para a rede neural. Para ter um treinamento mais estável,
se é utilizado duas redes, onde a rede alvo é atualizada usando o parâmetro
de polyak $\rho$ após o fim de uma época $e$.
O algoritmo do \textit{Deep Q-learning Network}
pode ser vista no Algoritmo \ref{Deep:Q-learning:}

\begin{algorithm}[H]
    %% \SetLine
    Parâmetros dos algoritmo:
    importância de um paço $\alpha \in (0,1]$,
    taxa de aprendizado $\gamma \in (0,1]$,
    polyak $\rho \in (0,1]$,
    capacidade da memória $C \in \mathbb{N}$ 


    inicialize os pesos $\theta$ da rede neural que aproxima a função $Q$ com valores aleatórios

    inicialize os pesos $\theta_{\text{target}}  \leftarrow \theta$ rede alvo $Q_{\text{target}}$

    \Para {$e \leftarrow 0$ \Ate $E$} {
        inicialize a sequencia $s_0$

        inicialize a sequência preprocessada $\phi_0 \leftarrow \phi(s_0)$ 
        

        \Para {$j \leftarrow 0$ \Ate $n$} {
            Escolha a ação $a_t$ do estado processado $phi_t$ usando alguma regra derivada de $Q$,
            por exemplo a $\epsilon$-greedy

            Realize a ação $a_t$ e observe a recompensa $r_t$ e o nova imagem $x_{t+1}$
            
            $s_{t+1}  \leftarrow (s_t, a_t, x_{t+1})$   

            preprocesse a sequencia $\phi_{t+1} \leftarrow \phi(t+1)$
            
            armazene a transição $(\phi_t,a_t,r_t,\phi(t+1))$ na memória $M$

            \Se {$K \ge$ total de transições armazenadas}{
                Selecione $K$ amostras para formar um minibatch de transições

                inicialize o conjunto $y$ de valores desejados
    
                \Para {$k \leftarrow =0$ \Ate $K$}{
    
    
                    \eSe {$s_k$ é terminal}{
                    
                        $y_k \leftarrow r_k$
                      }
                      {
                        $y_k \leftarrow r_k + \gamma \max_aQ(\phi_k,a_k)$
                      }
                    
                }
    
                atualize os pesos $\theta$ baseado no erro $\frac{1}{|K|} \sum_{b =0}^{K}(y -Q(\phi_b,a_b))^2$

        
            }

            \Se {$s_{t+1}$ é terminal}{
                
                $j \leftarrow N_p$  \textbf{ fim do loop de paços}
              }

        }

        ao fim de uma época,atualize os pesos da rede alvo
        $\theta_{\text{target}}  \leftarrow \rho \theta_{\text{target}}  + (1-\rho) \theta$
        
    }
    \Retorna a rede neural $Q_{\text{target}}$
    \caption{Algoritmo Deep Q-learning}
    \label{Deep:Q-learning:}
\end{algorithm}

Como dito anteriormente A \textit{Deep Q-learning}, possuí o espaço das ações são discretas,
visando tornar o espaço de ações também continuas é acrescentando mais
uma rede neural para aproximar a política $\mu$,
desta forma temos um algoritmo baseado na equação \textit{Q-learning} que permite
aprender um sistema de entrada e saída contínua.
\begin{flalign} 
    A_t = \mu(S_t)\\
    Q(S_t,A_t) \leftarrow Q(S_t,A_t) + \alpha[R + \gamma  Q(S_{t +1},\mu(S_{t+1})) - Q(S_t,A_t)]
\end{flalign}
Enquanto temos uma rede neural $Q$ que busca encontrar a função ação-valor ótima $Q^*$, temos uma
nova rede neural que através do gradiente ascendente buscar encontrar a política $\mu$
que maximiza a recompensa da função valor aproximada $Q(S,\mu(S))$. O algoritmo do \textit{Deep Deterministic Policy Gradient}
pode ser vista no Algoritmo \ref{Deep Deterministic Policy Gradient:}. Ambos os
algoritmos \textit{Deep Q-learning} e \textit{Deep Deterministic Policy Gradient} 
foram retirados da documentação da \textit{OpenIA} \cite{SpinningUp2018}.

\begin{algorithm}[H]
    %% \SetLine
    Parâmetros dos algoritmo:
    importância de um paço $\alpha \in (0,1]$,
    taxa de aprendizado $\gamma \in (0,1]$,
    polyak $\rho \in (0,1]$,
    capacidade da memória $C \in \mathbb{N}$ 

    inicialize a memória $M$ com a capacidade $C$

    inicialize os pesos $\theta$ da rede neural que aproxima a função $Q$ com valores aleatórios

    inicialize os pesos $\theta_\mu$ da rede neural que a aproxima da política $\mu(\phi)$ com valores aleatórios

    inicialize os pesos $\theta_{\text{target}}  \leftarrow \theta$ rede alvo $Q_{\text{target}}$

    inicialize os pesos $\theta_{\mu_{\text{target}}}  \leftarrow \theta_\mu$ rede alvo $\mu_{\text{target}}$


    \Para {$e \leftarrow 0$ \Ate $E$} {
        inicialize a sequencia $s_0$

        inicialize a sequência preprocessada $\phi_0 \leftarrow \phi(s_0)$ 
        

        \Para {$j \leftarrow 0$ \Ate $n$} {
            Escolha a ação $a_t$ do estado processado $phi_t$ usando a rede  $\mu(\phi) + \epsilon$, onde
            $\epsilon$ é um ruído que controla a exploração, $a_t \leftarrow \mu(\phi_t)  + \epsilon$ 

            Realize a ação $a_t$ e observe a recompensa $r_t$ e o nova imagem $x_{t+1}$
            
            $s_{t+1}  \leftarrow (s_t, a_t, x_{t+1})$   

            preprocesse a sequencia $\phi_{t+1} \leftarrow \phi(t+1)$
            
            armazene a transição $(\phi_t,a_t,r_t,\phi(t+1))$ na memória $M$

            \Se {$K \ge$ total de transições armazenadas}{
                Selecione $K$ amostras para formar um minibatch de transições

                inicialize o conjunto $y$ de valores desejados
    
                \Para {$k \leftarrow =0$ \Ate $K$}{
    
    
                    \eSe {$s_k$ é terminal}{
                    
                        $y_k \leftarrow r_k$
                      }
                      {
                        $a_d = \mu_{\text{target}}(\phi_k)$
    
                        $y_k \leftarrow r_k + \gamma Q_{\text{target}}(\phi_k,a_d)$
                      }
                    
                }

                atualize os pesos $\theta$ baseado no erro  $\frac{1}{|K|} \sum_{b =0}^{K} (y -Q(\phi_b,\mu(\phi_b)))^2$
                
                atualize os pesos $\theta_\mu$ baseado no gradiente ascendente:
                $\nabla_{\theta_\mu} \frac{1}{|K|} \sum_{b =0}^{K} Q(\phi_b,\mu(\phi_b))$
            
            }

           
        }

        atualize os pesos da rede alvo $\theta_{\text{target}}  \leftarrow \rho \theta_{\text{target}}  + (1-\rho) \theta$

        atualize os pesos da rede alvo $\theta_{\mu_{\text{target}}}  \leftarrow \rho \theta_{\mu_{\text{target}}}  + (1-\rho)\theta_\mu$

        
    }
    \Retorna as redes neurais $Q_{\text{target}}$, $\mu_{\text{target}}$
    \caption{Algoritmo Deep Deterministic Policy Gradient}
    \label{Deep Deterministic Policy Gradient:}
\end{algorithm}


\section{Conclusão da revisão bibliográfica}
Os artigos demostram que para um problema simples como posicionar ou estabilizar
um robô de acionamento diferencial em um ambiente sem obstáculos ou com
obstáculos simples, as soluções com aprendizado de máquina por reforço
envolvendo variantes de \textit{Q-learning} visando substituir as
leis clássicas de controle não são interessantes, pois usam muito mais
memória, processamento e apresentam resultados equiparáveis quando não
piores que as soluções utilizando leis clássicas de controle. 

% Cap. 4 - Problema
%%
%% Capítulo 4: Desenvolvimento
%%

\mychapter{Desenvolvimento}
\label{Cap:Desenvolvimento}

O principal objetivo desse trabalho é criar um sistema de controle cinemático para
um Andador inteligente, sendo este  sistema de controle cinemático pensado para requerer
pouca memória e processamento. Também visa-se estudar e aplicar algoritmos
de aprendizagem de máquina de modo a gerar um modelo cinemático do robô.
O modelo cinemático é o maior desafio deste trabalho, pois como será visto
mais adiante foi utilizado uma abordagem que mescla a solução analítica com
algoritmos de aprendizado de máquina para gerar um modelo de um robô com
acionamento diferencial. Outra contribuição deste trabalho é o modelo do
robô simulado, onde o controlador e os modelos cinemáticos foram avaliados.
O desenvolvimento do  sistema de controle cinemático foi dividindo em quatro
partes. A primeira é a construção da simulação do robô.
Segunda foi o desenvolvimento do algoritmo de coleta de dados necessária
para o treinamento dos modelos cinemáticos. Terceira foi  modelado
duas redes neurais de modo que seus parâmetros se traduzissem nos parâmetros
da cinemática. Quarta e ultima parte conta o desenvolvimento
do algoritmo do pré-processamento dos dados que transforma o conjunto
de dados coletados em um conjunto de dados prontos para o treinamento. 


\section{construção do robô em um ambiente simulado}
O simulador utilizado para a construção do robô foi o CoppeliaSim
\cite{rooban2021coppeliasim}, antigamente conhecido como V-REP.
Uma das contrições deste trabalho foi a criação de um cliente 
\textit{zmqRemoteApi} para a linguagem de programação \textit{Rust}
que se comunica com o simulador. \textit{Rust} foi adotada pois
é capaz de produzir um código tão performático quanto C/C++, além
de possui um gerenciador de pacotes padrão que facilita o desenvolvimento
futuro de novas aplicações e reutilização de códigos. O ambiente
simulado possui o formato quadrado com um lado $l$ de 5 metros.
Além do robô, o ambiente possui um alvo, o qual o robô deve
dirigir-se. O alvo é um objeto que pode ser movido com mouse. 
Uma imagem do cenário completo pode ser visualizado na figura
\ref{fig:cenario:completo}

\begin{figure}[H]
    \label{fig:cenario:completo}
    \centering
    \includegraphics[height=5.5cm]{figuras/robo_simulado_1.png}%
    \hspace{1cm}
    \includegraphics[height=5.5cm]{figuras/visao_cima.png}
    \caption{Andador inteligente simulado}
\end{figure}

\begin{figure}[H]
    \label{fig:dinamica:robo}
    \centering
    \includegraphics[height=5.5cm]{figuras/robo_dinamica_1.png}
    \hspace{1cm}
    \includegraphics[height=5.5cm]{figuras/robo_dinamica_2.png}
    \caption{Dinâmica do robô}
\end{figure}

Como dito anteriormente, o robô possui um acionamento diferencial.
CoppeliaSim permite configurar os atuadores no modo controle de
velocidade. neste modo, o cliente \textit{zmqRemoteApi} é capaz de enviar
um sinal em radianos por segundo, o qual é aplicado instantaneamente.
O modelo conta com sensores para odometria como giroscópio e acelerômetro
e uma câmera RGB-D do modelo Kinect, no entanto os sensores não foram
utilizados neste trabalho.
Foram utilizados funções do próprio simulador para coletar posição e 
orientação do robô em relação ao referencial global.
Segundo a documentação do CoppeliaSim na parte da dinâmica,
devemos priorizar a montagem do robô com peças primitivas do próprio
simulador. Portanto regiões que deveriam ser mais arredondadas foram
aproximadas por retângulos de modo que seja possível ser utilizada uma
primitiva. Para a detecção de colisão ou para as detecções realizadas
pelos sensores,como: câmeras e sonars, o modelo percebido é o da figura
\ref{fig:cenario:completo}. Para o comportamento do robô a colisões e toda dinâmica da simulação
o modelo percebido é o da figura \ref{fig:dinamica:robo}. O robô simulado é a junção dos dois
modelos, onde durante a simulação o modelo da dinâmica está configurado para ser invisível.
Já o modelo de percepção é um objeto 3D sem dinâmica a qual sua posição e orientação
está ligada a posição e orientação do modelo dinâmico. Quando o modelo dinâmico se
move pelo ambiente simulado, o modelo visual acompanha. 


\section{Coleta de dados para o modelo cinemático}
Para coletar dados que serão utilizados pelo sistema de aprendizado de máquina
foi criado um algoritmo que faz o robô se mover aleatoriamente pelo cenário.
Durante o movimento aleatório era coletado dados do robô.
Os dados eram: a posição, orientação do robô em relação ao referencial global e
as velocidades angulares das rodas. O pseudo código pode ser observado no Algoritmo \ref{coleta:de:dados:}

\begin{algorithm}[H]
    \label{coleta:de:dados:}
    
    \Entrada{número de amostras $N_a$, número de passos contínuos $K$ }
    %% \SetLine
    
    inicialize a conexão SIM  com o simulador

    inicialize a memória $M$

    mova o robô R para a origem e com uma orientação aleatória $\theta$,
    por meio da conexão SIM

    \Para {$E \leftarrow 0$ \Ate $N_a$} {
        leia a posição $x_1$,$y_1$  e orientação $\theta_{1}$ do robô,
        referente a origem, por meio da conexão SIM
        
        leia o tempo $t_1$ da simulação,por meio da conexão SIM
        
        gere as velocidades das rodas $\phi_l$,$\phi_r$ aleatoriamente,
        entre [0,$V_{MAX}$]
        
        envie  $\phi_l$,$\phi_r$, para o robô simulado pela conexão SIM

        permita que a simulação ocorra por 50 ms 

        leia a posição $x_2$,$y_2$  e orientação $\theta_{2}$ do robô,
        referente a origem, por meio da conexão SIM

        leia o tempo $t_2$ da simulação.por meio da conexão SIM

            \Se {$E$ é múltiplo de $K$}{
                
                mova o robô R para a origem e com uma orientação aleatória $\theta$,
                por meio da conexão SIM
            }
        
        armazene em $M$ os valores  $(x_1,y_1,\theta_{1},t_1,\phi_l,\phi_r),(x_2,y_2,\theta_{2},t_2)$
        
    }

    armazene $M$ em um arquivo
    
    \caption{Algoritmo de Coleta de dados}
    
\end{algorithm}

O parâmetro $K$ do algoritmo foi criado para que o robô não bata na parede
da simulação, ele foi encontrado fazendo um teste empírico, mostrando que
após 20 passos  com o robô caminhando em linha reta, o robô bate na parede,
por tanto nos nossos testes $K = 18$. Um princípio foi adotado na coleta
de dados: o robô deve-se mover lentamente em todas a direções.
Essa estrategia é para evitar quebrar ou danificar o robô durante a coletada
de dados. Com esse principio em mente foi adotado uma velocidade linear
máxima para o robô de 16 centímetros por segundo. Na pratica isso significa
que será gerado dois números aleatórios de 0 até 2,
onde 2 é o valor da velocidade angular maxima $V_{MAX}$ das rodas.
As leituras de posição do robô retorna valores entre 
-2,5 até 2,5 metros.
A leitura da orientação do robô retorna um valor em radianos que varia entre
$-\pi$ até $\pi$.


\section{Modelagem de parâmetros de redes neurais artificiais}
Atualmente \textit{Frameworks} de aprendizado máquina supervisionado
têm evoluído bastante. Uma das principais técnicas que revolucionou a
área de aprendizado de máquina é uma estrutura de dados chamada grafo
de computação. Ele é o um grafo direcionado e acíclico de operações com
tensores, onde um nó pode ser um tensor ou uma função que opera sobre
tensores. A figura \ref{fig:grafo:computacional} é um exemplo de
grafo de computação que foi retirada do livro \cite{chollet2021deep},
onde $x$ e $y_{true}$ são as variáveis de entrada, e w,b são
os parâmetros da rede neural artificial. O grafo computacional
representa um modelo: $W \times x + B$ para uma função erro $loss$ qualquer.

\begin{figure}[H]
    \label{fig:grafo:computacional}
    \centering
    \includegraphics[scale=0.5]{figuras/grafo_computacional.png}
    \caption{grafo de computação}
\end{figure}

Como dito no capítulo \ref{Cap:Teoria}, cada uma das duas rodas segue as equações:
\begin{align}
    \frac{1}{r}
    \begin{bmatrix}
        \sin(\alpha + \beta) &  -\cos(\alpha + \beta) & -l\cos(\beta) \
    \end{bmatrix}
    \dot{\xi}
    = \phi \\
    \begin{bmatrix}
        \cos(\alpha + \beta) &  \sin(\alpha + \beta) & l\sin(\beta) \
    \end{bmatrix}
    \dot{\xi}
    = 0 
\end{align}
onde $\dot{\xi}$ é o vetor de velocidades do robô. Sabendo que o robô possui duas rodas de acionamento diferencial
podemos obter a seguinte matriz $W$ :
\begin{flalign}\label{eq:matriz:cinematica}
    W =
    \begin{bmatrix}
        \frac{\sin(\alpha_{1,1} + \beta_{1,1})}{r_1} &  \frac{-\cos(\alpha_{1,1} + \beta_{1,1})}{r_1} & \frac{-l\cos(\beta_{1,1})}{r_1} \\
        \frac{\sin(\alpha_{2,1} + \beta_{2,1})}{r_2} &  \frac{-\cos(\alpha_{2,1} + \beta_{2,1})}{r_2} & \frac{-l\cos(\beta_{2,1})}{r_2}\\
        \cos(\alpha_{1} + \beta_{1}) &  \sin(\alpha_{1} + \beta_{1}) &  l_1\sin(\beta_{1}) \\
        \cos(\alpha_{2} + \beta_{2}) &  \sin(\alpha_{2} + \beta_{2})  &  l_2\sin(\beta_{2})\\
    \end{bmatrix} 
\end{flalign}
onde $\alpha_1$,$\beta_1$,$l_1$ e $r_1$ são os parâmetros da roda esquerda
e $\alpha_2$,$\beta_2$,$l_2$ e $r_2$ são os parâmetros da roda direita.
Portanto foi modelado um grafo computacional que seja equivalente
a matriz $W$ da equação \ref{eq:matriz:cinematica}.
O grafo de computação com essas operações é representado por essas equações:

\begin{align}\label{eq:grafo:computacional}
    \gamma = \alpha + \beta \\
    \cos_{\gamma} = \cos(\gamma) \\
    \sin_{\gamma} = \sin(\gamma) \\
    l_{\phi} = \frac{-l\cos(\beta)}{r} \\
    l_{0} = l\sin(\beta) \\
    W_{\phi_1} = \frac{\sin_{\gamma}}{r} \\
    W_{\phi_2} = \frac{-\cos_{\gamma}}{r} \\
    W_{\phi} = \textbf{tensor\_concat\_h}(W_{\phi_1},W_{\phi_2}, l_{\phi})\\
    W_{0} = \textbf{tensor\_concat\_h}(\cos_{\gamma},\sin_{\gamma},  l_{0}) \\
    W = \textbf{tensor\_concat\_v}( W_{\phi}, W_{0} ) \\
    y_{\textbf{pred}} =  W \times  \dot{\xi} 
\end{align}
onde cada variável é um tensor, e a
função \textbf{tensor\_concat\_h} junta os tensores horizontalmente:
\begin{align}
    \alpha = 
    \begin{bmatrix}
         \alpha_{1,1} \\
         \alpha_{2,1}
    \end{bmatrix}
    \\
    \beta =
    \begin{bmatrix}
        \beta_{1,1} \\
        \beta_{2,1}
   \end{bmatrix}\\
   \textbf{tensor\_concat\_h}( \alpha, \beta ) =
   \begin{bmatrix}
    \alpha_{1,1} && \beta_{1,1} \\
    \alpha_{2,1} &&  \beta_{2,1} \\
\end{bmatrix}
\end{align}
e \textbf{tensor\_concat\_v} junta os tensores verticalmente:
\begin{align}
    \alpha = 
    \begin{bmatrix}
         \alpha_{1,1} \\
         \alpha_{2,1}
    \end{bmatrix}
    \\
    \beta =
    \begin{bmatrix}
        \beta_{1,1} \\
        \beta_{2,1}
   \end{bmatrix}\\
   \textbf{tensor\_concat\_v}( \alpha, \beta ) =
   \begin{bmatrix}
    \alpha_{1,1} \\
    \alpha_{2,1} \\
    \beta_{1,1} \\
    \beta_{2,1} \\
\end{bmatrix}
\end{align}
Perceba que $W_{\phi}$ representa as equações das rodas, onde
o valor desejado são as velocidades angulares $\phi_1,\phi_2$ das rodas,
ou seja, $W_{\phi} \times \dot{\xi} = [\phi_{1},\phi_2]^{T} $
\begin{align}
    W_{\phi} = 
    \begin{bmatrix}
        \frac{\sin(\alpha_{1,1} + \beta_{1,1})}{r_1} &  \frac{-\cos(\alpha_{1,1} + \beta_{1,1})}{r_1} & \frac{-l\cos(\beta_{1,1})}{r_1} \\
        \frac{\sin(\alpha_{2,1} + \beta_{2,1})}{r_2} &  \frac{-\cos(\alpha_{2,1} + \beta_{2,1})}{r_2} & \frac{-l\cos(\beta_{2,1})}{r_2}
    \end{bmatrix}
\end{align}
e $W_0$ representa a parte das equações das rodas cujo o valor desejado
é 0, ou seja, $W_{\phi} \times \dot{\xi} = [0,0]$.
\begin{align}
    W_{0} = 
    \begin{bmatrix}
        \cos(\alpha_{1} + \beta_{1}) &  \sin(\alpha_{1} + \beta_{1}) &  l_1\sin(\beta_{1}) \\
        \cos(\alpha_{2} + \beta_{2}) &  \sin(\alpha_{2} + \beta_{2})  &  l_2\sin(\beta_{2})
    \end{bmatrix}
\end{align}
Perceba que concatenando verticalmente a matriz $W_{\phi}$ e $ W_{0}$ temos a matriz $W$
da equação \ref{eq:matriz:cinematica}. 
Os tensores: $\alpha =[\alpha_1,\alpha_2]$,
$\beta=[\beta_1,\beta_2]$ , $l=[l_1,l_2]$ e $r=[r_1,r_2]$
da sequência de equações \ref{eq:grafo:computacional}.
São esses parâmetros que são encontrados por meio dos algoritmos RMSprop
e Backpropagation. Na prática
estamos fazendo uma regressão não linear
para encontrar os parâmetros de uma transformação linear, então podemos também
pensar em grafo de computacional mais simples e semelhante a
figura \ref{fig:grafo:computacional} mas sem o parâmetro $b$.
Ou seja, podemos encontrar a matriz $W_l$:
\begin{align}
    W_l = 
    \begin{bmatrix}
        w_{1,1} &  w_{1,2} &  w_{1,3} \\
        w_{2,1} &  w_{2,2} &  w_{2,3}
    \end{bmatrix}
\end{align}
onde $W_l \times \dot{\xi} = [\phi_{1},\phi_2]^{T}$.
Ambos os modelos foram treinados e foram discutidos na
sessão de experimentos e resultados.


\section{pré-processamento dos dados}
A coleta de dados nos fornece dados de posição, orientação, velocidade angular
das rodas do robô e tempo de simulação. Portanto temos que transformar os dados
de posição e orientação e tempo, em informação de velocidade linear e angular
do robô. Sabendo que a velocidade linear está no referencial global, então
temos que mudar o referencial para o do robô.
O pseudo código do pré-processamento pode ser observado no Algoritmo \ref{pre:processamento:}

\begin{algorithm}[H]
    \label{pre:processamento:}
    
    \Entrada{$F$:arquivo resultante da coleta de dados }
    %% \SetLine
    
    $(x_1,y_1,\theta_{1},t_1,\phi_l,\phi_r),(x_2,y_2,\theta_{2},t_2) \leftarrow$ leitura da coleta dados dos instantes de
    tempo $t_1$ e $t_2$

    $\Delta x \leftarrow x_2 - x_1$
    
    $\Delta y \leftarrow y_2 - y_1$

    $\Delta t \leftarrow t_2 - t_1$

    $\Delta \theta \leftarrow \theta_2 - \theta_1$

    \Se {$\Delta \theta > \pi$ }{
                $\Delta \theta \leftarrow \Delta \theta -2\pi $
    }

    \Se {$\Delta \theta < -\pi$ }{
                $\Delta \theta \leftarrow \Delta \theta +2\pi $
    }

    $\text{vel}_{\text{linear}_G} \leftarrow \frac{(\Delta x,\Delta y)}{\Delta t}$

    $\text{vel}_{\text{angular}} \leftarrow \frac{\Delta \theta}{\Delta t}$



    \Para {$i \leftarrow 0$ \Ate $N_a$} {
       
        $
        R(\theta) \leftarrow 
        \begin{bmatrix}
            cos(\theta_{1,i})  & sin(\theta_{1,i})\\
            -sin(\theta_{1,i}) & cos(\theta_{1,i})\\
        \end{bmatrix}
        $

        $\text{vel}_{\text{linear}_{R,i}} \leftarrow R(\theta) \times \text{vel}_{\text{linear}_{G,i}}$
    }
    
    entrada\_modelo\_cinemático $\leftarrow (\text{vel}_{\text{linear}_{R}},\text{vel}_{\text{angular}})$

    
    \eSe {for a entrada do modelo simples: $\phi=W \times \dot{\xi}$}{
                    
        saída\_modelo\_cinemático $\leftarrow (\phi_l,\phi_r)$
      }
      {
        saída\_modelo\_cinemático $\leftarrow (\phi_l,\phi_r,0,0)$
      }

   

      \Retorna entrada\_modelo\_cinemático, saída\_modelo\_cinemático
    
    \caption{pré-processamento dos dados}
    
\end{algorithm}

Perceba que após a leitura dos dados de entrada é feita uma aproximação da
velocidade angular e linear do robô através da variação angular sobre variação de tempo e
variação de deslocamento sobre variação de tempo.
Foi observado que deixar variação da orientação do robô entre
$-\pi$ e $\pi$ torna o conjunto de dados mais fácil de ser aprendido.
Portanto ao calcular a variação da orientação do robô é selecionado
sempre a menor diferença angular resultante $\Delta \theta$.  
É importante salientar que $N_a$
é um número de amostras coletadas, este número pode ser recuperado através
do tamanho de um dos vetores: $x_1,y_1,\theta_{1},t_1,\phi_l,\phi_r,x_2,y_2,\theta_{2},t_2$.
Como dito anteriormente, foram testados dois modelos, um modelo mais complexo
que envolve modelagem dos parâmetros da rede através das equações das rodas
e outro modelo mais simples que busca encontrar uma transformação linear
para os conjunto de dados. Quando é realizado o treinamento do modelo
cujo o grafo computacional é mais complexo.
São adicionados mais dois vetores de zeros á saída do modelo cinemático, zeros
que são esperados pelas equações das rodas. 

% Cap. 5 - Implementação
% %%
%% Capítulo 4: Figuras, gráficos e tabelas
%%

\mychapter{Implementação}
\label{Cap:Implementacao}
Lorem ipsum dolor sit amet, consectetur adipiscing elit. Morbi tristique, orci mollis tincidunt dignissim, purus lectus molestie odio, vitae pharetra nisi sapien et justo. Fusce consequat et elit condimentum tincidunt. In eu venenatis tortor, quis lobortis tellus. Mauris tincidunt gravida ante. Pellentesque ante elit, lacinia interdum bibendum sed, cursus non orci. Ut et diam nec justo efficitur tincidunt. Fusce convallis facilisis varius. Integer tempor hendrerit maximus.

\begin{algorithm}
%% \SetLine
\Entrada{$x$: vetores de valores; $y$ = $L(x)$; $p$: valor de entrada a ser calculado }
\Saida{$s$ = $L(p)$}
$n \leftarrow \mathtt{comprimento}(x)$\;
$s \leftarrow 0$\;
\Para {$i=1$ \Ate $n$} {
	$L \leftarrow 1$\;
	\Para {$j=1:1:n$} {
		\Se{$i \neq j$} {
			$L \leftarrow L* \left( \dfrac{p-x[j]}{x[i]-x[j]} \right) $
		}
	}
	$s \leftarrow s + L*y[i]$\;
}
\Retorna $s$\;
\caption{Algoritmo para interpolação de Lagrange.}
\label{algo:1}
\end{algorithm}

\begin{algorithm}
%% \SetLine
\Entrada{$a$: valor inicial; $b$: valor final; $n$: número de subintervalos (deve ser múltiplo de 2)  }
\tcc{A função a ser integrada é definida em uma função denominada \texttt{f}, fora do escopo deste algoritmo.}
\Saida{$I$ = integral de \texttt{f} entre $a$ e $b$}
$h \leftarrow$ $\dfrac{b-a}{n}$\;
$x[1] \leftarrow a$\;
$y[1] \leftarrow f(a)$\;
$I \leftarrow 0$\;
$k \leftarrow 2$\;
\Enqto {$k <= n$} {
	$x[i] \leftarrow x[i-1] + h$\;
	$y[i] \leftarrow f(x[i])$\;
	\eSe{$i \% 2 = 0$} {
		$I \leftarrow I + 4*y[i]$\;
	}
	{
		$I \leftarrow I + 2*y[i]$\;
	}
	$k = k+1$\;
}
$x[n+1] \leftarrow b$\;
$y[n+1] \leftarrow f(x[i+1])$\;
$I \leftarrow I + \dfrac{h}{3}*(I + y[n+1])$\;
\Retorna $I$\;
\caption{Algoritmo para a integração pelo primeiro método de Simpson.}
\label{algo:2}
\end{algorithm}



\section{Nulla molestie libero sed}
\label{Sec:expressoesMatematicas}

\begin{eqnarray} \label{eq:PDF:RSR}
  p \left( \gamma \right) & = & \frac{1}{2} \sqrt{\frac{M}{\gamma \bar{\gamma}_{b}}} \frac{1}{ \prod_{i=1}^M {\sqrt{\tilde{\gamma}_i}}}
  \int_0^{\sqrt{M \delta}} \int_0^{\sqrt{M \delta} - r_M } \cdots
  \int_0^{\sqrt{M \delta} - \sum_{i = 3}^M {r_i } } \nonumber \\
  & & p \left( {\frac{\sqrt{M \delta} - \sum_{i = 2}^M {r_i }}{\sqrt{\tilde{\gamma}_1}} ,
  \frac{r_2}{\sqrt{\tilde{\gamma}_2}} , \ldots ,\frac{r_M}{\sqrt{\tilde{\gamma}_M}} } \right)
  \, dr_2 \cdots dr_{M-1} \, dr_M
\end{eqnarray}
% sem linha em branco
ou:
% sem linha em branco
\begin{equation} \label{eq:TrCGI}
  T(r) = \frac{1}{f_m}
  \left( \frac{\pi}{2} \sum_{i=1}^M
  {\tilde{r}_i^2 \dot{\varsigma}_i^2}\right)^{-1/2}
  \frac
  {\begin{array}{ll}
  \int_0^{\rho \sqrt{M}} \int_0^{\rho \sqrt{M} - r_M } \cdots
  \int_0^{\rho \sqrt{M} - \sum_{i = 3}^M {r_i } } \int_0^{\rho \sqrt{M} -
  \sum_{i = 2}^M {r_i } }  \\
  p \left( {\frac{r_1}{\tilde{r}_1} ,
  \frac{r_2}{\tilde{r}_2} , \ldots ,\frac{r_M}{\tilde{r}_M} } \right)
  \, dr_1 \, dr_2 \cdots dr_{M-1} \, dr_M \\ \end{array}}
  {\begin{array}{ll}
  \int_0^{\rho \sqrt{M}} \int_0^{\rho \sqrt{M} - r_M } \cdots
  \int_0^{\rho \sqrt{M} - \sum_{i = 3}^M {r_i } } \\
  p \left( {\frac{\rho \sqrt{M} - \sum_{i = 2}^M {r_i }}{\tilde{r}_1} ,
  \frac{r_2}{\tilde{r}_2} , \ldots ,\frac{r_M}{\tilde{r}_M} } \right)
  \, dr_2 \cdots dr_{M-1} \, dr_M \\ \end{array}}
\end{equation}


\begin{equation}
y = g(x) = g(x_{PO}) + \left.\frac{dg}{dx}\right|_{x=x_{PO}}
\frac{(x-x_{PO})}{1!} + \left.\frac{d^2g}{dx^2}\right|_{x=x_{PO}}
\frac{(x-x_{PO})^2}{2!} + \cdots
\label{Eq:Taylor}
\end{equation}



% Cap. 6 - Experimentos e Resultados
%%
%% Capítulo 4: Figuras, gráficos e tabelas
%%

\mychapter{Experimentos e Resultados}
\label{Cap:ExperimentosResultados}

Lorem ipsum dolor sit amet, consectetur adipiscing elit. Morbi tristique, orci mollis tincidunt dignissim, purus lectus molestie odio, vitae pharetra nisi sapien et justo. Fusce consequat et elit condimentum tincidunt. In eu venenatis tortor, quis lobortis tellus. Mauris tincidunt gravida ante. Pellentesque ante elit, lacinia interdum bibendum sed, cursus non orci. Ut et diam nec justo efficitur tincidunt. Fusce convallis facilisis varius. Integer tempor hendrerit maximus.

Donec pulvinar, libero semper pretium fringilla, arcu ligula iaculis urna, id faucibus ante massa a augue. Ut iaculis fermentum convallis. Vestibulum ac porttitor urna, sit amet dignissim dui. Curabitur sollicitudin tincidunt risus quis vulputate. In in leo ut mi mattis tempor sed non nunc. Sed et velit nibh. Duis mollis lectus risus, ut vestibulum purus semper eu. Donec rhoncus ex sed scelerisque dictum. Praesent sit amet neque accumsan, euismod sapien eget, tincidunt magna. Etiam nec diam ac elit feugiat aliquet. Pellentesque et neque metus. Nulla molestie libero sed ullamcorper mollis. In a risus quis felis eleifend dapibus at mollis odio. Curabitur nisl nibh, placerat ac quam nec, vulputate blandit mauris.


% Cap. 7 - Conclusão
%%
%% Capítulo 5: Conclusões
%%

\mychapter{Conclusão}
\label{Cap:Conclusao}

Este trabalho visou aplicar e avaliar sistemas de aprendizado de máquina
na construção de um controlador para um robô móvel de acionamento diferencial
que ajuda pessoas com mobilidade reduzida. Inicialmente foi feito uma
revisão bibliográfica onde concluímos que sistemas de aprendizado de
máquina que substitui um controlador clássico não é uma boa abordagem,
considerando um cenário simples como foi apresentado neste trabalho.
Portanto foi utilizado o controlador Vieira.
Podemos concluir que uma vez treinado a abordagem utiliza pouca memória,
sendo a coleta de dados o momento mais crítico. Por fim podemos concluir
que resolver o problema da forma clássica, ou seja resolvendo a cinemática
do modelo e criando um controlador clássico é a forma mais simples
de resolver o problema, no entanto com apenas com o conhecimento das equações,
podemos modelar um grafo computacional que encontre os parâmetros das
equações de modo que a solução equivale ao modelo analítico, um fato
interessante que precisa ser melhor investigado.

% Referências bibliogáficas (geradas automaticamente)
% Aqui, o comando \phantomsection é utilizado para auxiliar o pacote "hyperref" a fazer a
% referência correta dos links das referências bibliográficas com a página respectiva.
% Caso seja tirado, o "hyperref" irá apontar o link das referências bibliográficas para a
% última subseção da conclusão.
\phantomsection
\addcontentsline{toc}{chapter}{Referências bibliográficas}
\bibliography{bibliografia/bibliografia}

% \appendix

%Apêndice A
% %%
%% Capítulo 5: Conclusões
%%

\mychapter{Informações adicionais}
\label{Cap:apendice}

Lorem ipsum dolor sit amet, consectetur adipiscing elit. Morbi tristique, orci mollis tincidunt dignissim, purus lectus molestie odio, vitae pharetra nisi sapien et justo. Fusce consequat et elit condimentum tincidunt. In eu venenatis tortor, quis lobortis tellus. Mauris tincidunt gravida ante. Pellentesque ante elit, lacinia interdum bibendum sed, cursus non orci. Ut et diam nec justo efficitur tincidunt. Fusce convallis facilisis varius. Integer tempor hendrerit maximus.

Donec pulvinar, libero semper pretium fringilla, arcu ligula iaculis urna, id faucibus ante massa a augue. Ut iaculis fermentum convallis. Vestibulum ac porttitor urna, sit amet dignissim dui. Curabitur sollicitudin tincidunt risus quis vulputate. In in leo ut mi mattis tempor sed non nunc. Sed et velit nibh. Duis mollis lectus risus, ut vestibulum purus semper eu. Donec rhoncus ex sed scelerisque dictum. Praesent sit amet neque accumsan, euismod sapien eget, tincidunt magna. Etiam nec diam ac elit feugiat aliquet. Pellentesque et neque metus. Nulla molestie libero sed ullamcorper mollis. In a risus quis felis eleifend dapibus at mollis odio. Curabitur nisl nibh, placerat ac quam nec, vulputate blandit mauris.

Vestibulum neque lacus, fringilla a urna quis, egestas tincidunt orci. Phasellus rutrum elit at mauris feugiat, non egestas tortor dictum. Aliquam faucibus, velit eu aliquam fermentum, lacus sem pellentesque nibh, et tristique urna nibh eget purus. Morbi vitae felis posuere, rutrum turpis quis, rutrum lorem. Nam lacinia cursus neque sit amet fermentum. Mauris eu mauris diam. Quisque lacinia consequat quam, at convallis urna dignissim eget. Proin sit amet varius sem, vel facilisis purus. Sed non blandit sapien. Sed auctor venenatis nibh eu ornare. Quisque euismod urna ligula, quis vestibulum sapien lacinia vehicula. Aenean sit amet vehicula felis, a imperdiet tellus. Fusce eu sem urna. Integer placerat nibh in purus consectetur mattis. Aliquam erat neque, tincidunt ac porta sit amet, egestas non lacus.


\end{document}
