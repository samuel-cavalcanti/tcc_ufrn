%%
%% Capítulo 2: Regras gerais de estilo
%%

\mychapter{Fundamentação Teórica}
\label{Cap:Teoria}
Este trabalho envolve a cinemática de um robô móvel,
algoritmos de aprendizado de máquina e
a implementação de um controlador estabilizante
que utiliza as leis de controle. Portanto esta seção
visa explicar as partes fundamentais destas areas do conhecimento,
que foram utilizada neste trabalho. Primeiro abordaremos o problema
da cinemática focando em robôs moveis de acionamento diferencial,
segundo abordaremos o aprendizado de máquina supervisionado
relacionando o tema com a cinemática, terceiro explicaremos o
funcionamento do controlador cinemático estabilizante criado
por Federico Vieira \cite{vieira2006controle}.

\section{O problema da cinemática de um robô móvel}
Um robô de acionamento diferencial, onde as velocidades
angulares das $n$ rodas são: $\phi_0,\phi_1,\phi_2,...,\phi_n$.
Uma cinemática direta $f$ de um robô móvel é a uma função que mapeia as
velocidades
angulares das rodas para a velocidade linear $v$ e velocidade angular $\omega$
do robô, ou seja, $f(\phi_0,\phi_1,\phi_2,...,\phi_n) \rightarrow v,\omega$. Já
a cinemática inversa $g$, mapeia a velocidade angular e linear do robô para
um
conjunto de velocidades angulares das rodas,
$g(v,\omega) \rightarrow  \phi_0,\phi_1,\phi_2,...,\phi_n$.
Cinemática é portanto um conjunto de regras que relaciona a velocidade
linear $v$ e angular $\omega$
com as velocidades das rodas $\phi$.

\begin{figure}[H]
    \label{fig:robo:movel:acionamento:diferencial}
    \centering
    \includegraphics[scale=0.9]{figuras/robo.png}%
    \caption{Robô móvel com acionamento diferencial}
\end{figure}

A cinemática de um robô móvel está fortemente relacionada a modelagem
das rodas do robô. Para um robô com duas rodas de acionamento diferencial
como mostrado na figura \ref{fig:robo:movel:acionamento:diferencial}
as equações que regem o movimento dele são:

\begin{equation}\label{eq:cinmeatica:1}
    \begin{bmatrix}
        \sin(\alpha_{1} + \beta_{1}) &  -\cos(\alpha_{1} + \beta_{1}) &  -l_1\cos(\beta_{1})\\
        \sin(\alpha_{2} + \beta_{2}) &  -\cos(\alpha_{2} + \beta_{2}) &  -l_2\cos(\beta_{2})\\
    \end{bmatrix}
    \begin{bmatrix}
        v_x \\
        v_y \\
        \omega\\
    \end{bmatrix}
    =
    \begin{bmatrix}
        r_0\phi_0 \\
        r_1\phi_1 \\
    \end{bmatrix}
\end{equation}


\begin{equation}\label{eq:cinmeatica:2}
    \begin{bmatrix}
        \cos(\alpha_{1} + \beta_{1}) &  \sin(\alpha_{1} + \beta_{1}) &  l_n\sin(\beta_{1}) \\
        \cos(\alpha_{2} + \beta_{2}) &  \sin(\alpha_{2} + \beta_{2})  &  l_n\sin(\beta_{2})\\
    \end{bmatrix}
    \begin{bmatrix}
        v_x \\
        v_y \\
        \omega\\
    \end{bmatrix}
    =
    \begin{bmatrix}
        0 \\
        0 \\
    \end{bmatrix}
\end{equation}

\begin{figure}[H]
    \label{fig:parametros:roda:fixa}
    \centering
    \includegraphics[width=10cm]{figuras/fixed_wheel_params.png}
    \caption{Parâmetros de uma roda fixa}
\end{figure}
onde $\alpha$, $\beta$, $l$,$r$ são propriedades das rodas do robô,
$v_x$,$v_y$ são as velocidade lineares ao longo dos eixos x,y.
É importante salientar que os parâmetros das
rodas são extraídos seguindo o referencial do robô $X_r$,$Y_r$ assim como
as velocidades.
\begin{figure}[H]
    \centering
    % \includegraphics[scale=0.9]{figuras/robo.png}
    \includegraphics[scale=0.28]{figuras/robo_coordenadas.png}
   
    \[
    \begin{bmatrix}
        v_x \\
        v_y \\
        \omega
    \end{bmatrix}
    =
    \begin{bmatrix}
        cos(\theta)  & sin(\theta) & 0 \\
        -sin(\theta) & cos(\theta) & 0 \\
        0 & 0 & 1 \\
    \end{bmatrix}
    \begin{bmatrix}
        v_{x_I} \\
        v_{y_I} \\
        \omega
    \end{bmatrix}
\]
\caption{Mudança de referencial}
\label{fig:mudanca:referencial}
\end{figure}



Caso a velocidade linear do robô for medida em outro
referencial, é necessário fazer uma rotação do referencial
medido para a orientação $X_r$,$Y_r$. Na figura \ref{fig:mudanca:referencial}
podemos observar um exemplo de mudança de referencial,
onde $v_{x_I}, v_{y_I}$, são as velocidades no referencial $X_I,Y_I$.
Tanto a figura \ref{fig:mudanca:referencial} e \ref{fig:parametros:roda:fixa}
foram retirados do livro \cite{siegwart2011introduction}

\begin{figure}[H]
    \centering
    \includegraphics[scale=0.4]{figuras/smart_walker.png}
    \caption{Andador robótico inteligente}
    \label{fig:andador:robotico:inteligente}
\end{figure}


No nosso caso, o robô a ser modelado possui quatro rodas. Duas delas são
rodas de acionamento diferencial e as outras duas rodas são rodas castores.
Uma imagem do robô pode ser vista na figura \ref{fig:andador:robotico:inteligente}.
Apesar das rodas castores, contribuírem para o movimento, elas não restringem
o movimento, portanto podemos entender o modelo cinemático do nosso robô
como sendo uma transformação linear $W \times X = Y$, onde $X$ é o vetor
de velocidades, e $W$ a matriz de valores constantes obtidas após resolver
as equações: \ref{eq:cinmeatica:1} e \ref{eq:cinmeatica:2}.

\begin{figure}[H]
    \[
    \begin{bmatrix}
        W_{11} &  W_{12} & W_{13} \\
        W_{21} &  W_{22} & W_{23} \\
    \end{bmatrix}
    \begin{bmatrix}
        v_x \\
        v_x \\
        \omega \\
    \end{bmatrix}
    =
    \begin{bmatrix}
        \phi_{\text{left}} \\
        \phi_{\text{right}} \\
    \end{bmatrix}
\]
    \caption{Cinemática inversa}
\end{figure}

Um fato muito importante sobre a cinemática do robô de acionamento diferencial
é que resolvendo a cinemática analaticamente vamos perceber que
$\alpha + \beta = 90^{\circ}$ e $\sin(\beta) = 0$, portanto a equação:

\begin{equation}
    v_x\cos(\alpha + \beta)  + v_y\sin(\alpha + \beta) +  \omega l_n\sin(\beta) = 0
\end{equation}

irá se traduzir em:
\begin{equation}
    v_y = 0
\end{equation}
ou seja, segundo a solução analítica não é possível que
o robô translade no eixo y.

\section{Aprendizado de máquina}
Usualmente quando programamos computadores para
resolver determinada tarefa, nós codificamos as regras e
executamos o programa com os dados de entrada e obtemos a resposta.
Esta é a abordagem clássica de se resolver um problema. Uma outra abordagem
é criar sistemas de aprendizado máquina, onde nós possuímos os
dados de entrada, as respostas e queremos que a máquina retorne
para nós quais são as regras que transformam os dados nas respostas
desejadas \cite{chollet2021deep}.

\begin{figure}[H]
    \centering
    \includegraphics[scale=0.7]{figuras/machine_learning_diagram.pdf}
    \caption{Aprendizado de máquina e Programação clássica}
\end{figure}

Existem três tipos de aprendizado de máquina: aprendizado por reforço,
aprendizado supervisionado e aprendizado não supervisionado.
Aprendizado por reforço é o aprendizado de como mapear situações
para ações de modo que maximize um sinal de recompensa
\cite{sutton2018reinforcement}. Aprendizado supervisionado é o
tipo de aprendizado onde a máquina busca extrair padrões nos conjuntos
de dados de entrada  e do conjunto de dados das respostas de modo que
transforme um dado de entrada na resposta desejada. Aprendizado não
supervisionado também busca encontrar padrões entre os conjuntos de
dados de entrada e resposta, mas sem ter uma resposta correta
\cite{trask2019grokking}. Neste trabalho foi utilizado algoritmos de
aprendizado de máquina supervisionado com objetivo de extrair um modelo de
cinemática inversa do robô andador inteligente.

\begin{figure}[H]
    \centering
    \includegraphics[scale=0.7]{figuras/aprendizado_cinemática_inversa.pdf}
    \caption{Aprendizado de máquina no calculo da cinemática inversa}
\end{figure}

Os algoritmos de aprendizado supervisionado utilizados neste trabalho
foram o Backpropagation e uma variação do algoritmo decida do gradiente, chamada
RMSprop.
Dado uma função $L(M)$, onde $L$ é uma função contínua e derivável,
a decida do gradiente é uma busca que visa encontrar
a melhor matriz $M$ que minimiza uma função $L(M)$, basicamente a
decida do gradiente parte do suposto que existe a sequência
$M_0$, $M_1$, $M_2$, ..., $M_i$, $M_{i+1}$ até $M_k$, onde quando
chegar até $M_k$, o ponto: $(L(M_k),M_k)$ será um ponto de mínimo da função $L$ e a transição
entre o $M_{i}$ até $M_{i+1}$, segue a seguinte regra:

\begin{equation}
    M_{i+1} = M_i -\alpha \nabla L(M_i)
\end{equation}

onde $\nabla L(W_i)$ é o gradiente da função $L$ em relação a $M$ aplicado
a $M_i$ e $\alpha$ é um número que varia de 0 até 1,
também conhecido com taxa de aprendizado. Uma variação desse algoritmo
bastante utilizado é chamado RMSprop, ele se inspira na ideia de momentum
da física  onde  é adicionado uma constante $m$ análoga a massa e uma grandeza
vetorial $v_{\text{vel}}$ análoga a velocidade, criando uma nova equação de decida:

\begin{figure}[H]
    \begin{align*}
        v_{\text{vel}_t} = \alpha v_{\text{vel}_{t-1}} + (1 - \alpha)\nabla L(M_i)^2 \\
        b_t = mb_{t-1} + \frac{\nabla L(M_i)}{\sqrt{v_{\text{vel}_t} + \epsilon}} \\
        M_{i+1} = M_i - b_t
    \end{align*}
    \caption{Equações do RMSprop}
\end{figure}

As variáveis $v_{\text{vel}_t}$ e  $b_t$, são inicializadas como zero, pode acontecer
que  $\sqrt{v_{\text{vel}_t}}$ seja zero ou bem próximo disso então para não ter uma equação
divida por zero é adicionado uma variável $\epsilon =10^{-8}$, deixando o algoritmo numericamente
mais estável. Em contexto de aprendizado supervisionado a função $L$ é chamada de função erro, que 
através do RMSprop os parâmetros dos modelo são guiados no sentido de minimizar o seu valor.
No entanto não se possui diretamente o gradiente de $L$ em relação aos parâmetros, para isso
é utilizado o  Backpropagation que encontra os gradientes a partir
de pontos $X$ e $Y_{\text{true}}$ coletados. O conjunto de pontos $X$ são
os dados de entrada
e o conjunto de pontos  $Y_{\text{true}}$ são as respostas desejadas.
Partindo de  uma definição de um modelo como por exemplo uma
transformação geométrica $Y_{\text{pred}}= A \times X + B$, e uma definição de uma função
erro como: $L=(Y_{\text{true}}- Y_{\text{pred}})^2$ podemos utilizar o algoritmo
Backpropagation para encontrar os gradientes da função $L$ em relação a $A$
e  $B$. Basicamente o Backpropagation executa automaticamente a regra da cadeia
para encontrar os gradientes e com os gradientes podemos executar o RMSprop,
ou seja, $A_{i+1} = A_i -b_{t_a}$ e $B_{i+1} = B_i -b_{t_b}$. Onde $b_{t_a}$ e $b_{t_b}$
são os valores de $b_{t}$ do RMSprop das respectivas matrizes $A$ e $B$.

\begin{figure}[H]
    \begin{align*}
        \frac{dL}{dA } = \frac{dL}{dY_{\text{pred}} } \frac{dY_{\text{pred}}}{dA}  \\
        \frac{dL}{dB } = \frac{dL}{dY_{\text{pred}} } \frac{dY_{\text{pred}}}{dB} 
    \end{align*}
    \caption{Regra da cadeia aplicada ao modelo: $Y_{\text{pred}}= A \times X + B$ }
\end{figure}

\section{Controlador estabilizante Vieira}
Controladores cinemáticos ou controladores de movimento, podem ser 
divididos em dois tipos: seguidores de trajetória e estabilizante.
Controladores seguidores trajetória produzem um perfil
de caminho sobre o tempo, a qual o controlador envia sinais de
velocidade linear e angular para o robô de modo que ele siga a
trajetória idealizada. 
A trajetória é feita a partir da posição,orientação atual do robô e uma posição,orientação desejada.
Controladores estabilizantes eles recebem como entrada um estado atual do robô e o estado desejado
e buscam enviar sinais de velocidade linear e angular com o objetivo
de minimizar o erro entre o estado atual e desejado. Neste trabalho
foi utilizado um controlador estabilizante criado por Frederico
\cite{vieira2006controle}, o controlador recebe como entrada a posição
$p_c$  no plano x, y, uma orientação $\theta_c$ atual do robô e uma
posição desejada $p_d$ e enviar sinais de velocidade linear $v$ e
angular $\omega$ com o objetivo de estabilizar o robô no ponto desejado.

\begin{figure}[H]
    \centering
    \includegraphics[scale=0.8]{figuras/controlador_viera.pdf}
    \caption{Bloco controlador Vieira}
\end{figure}

Internamente o controlador utiliza  dois controladores proporcionais
integrais derivativos (P.I.D). Um controlador é responsável pela
velocidade linear $v$
e outro controlador é responsável pela velocidade angular $\omega$. 
Um controlador P.I.D é definido pela a seguinte função de
transferência $C(s)$ no domínio continuo $s$
\begin{equation}
    C(s) = K_p + \frac{K_i}{s} + K_ds
\end{equation}
onde $K_p$,$K_i$,$K_d$ são constantes que podem ser adquiridas ou
através da experimentação ou uma analise matemática \cite{ogata2010modern}. Para gerar sinais
velocidades $v$ e $\omega$, o controlador Vieira funciona da seguinte
maneira. Primeiro é calculado o vetor posição $p_{\text{diff}}$:
\[
    p_{\text{diff}} = p_d - p_c 
\]
segundo são calculadas suas coordenadas polares $l$, $\alpha$.

\[
    l = \sqrt{p_{\text{diff}_x}^2 + p_{\text{diff}_y}^2}
\]
onde $p_{\text{diff}_x}$,$p_{\text{diff}_y}$ são as coordenadas x,y
do vetor $p_{\text{diff}}$.  Terceiro é calculado o angulo $\gamma$:
\[
    \alpha =  \arctan(\frac{ p_{\text{diff}_y}}{p_{\text{diff}_x}}) 
\]

\[
    \gamma =  \alpha - \theta
\]
Por fim o controlador P.I.D de velocidade
linear, busca enviar um sinal $v$ que faça tender o valor $l \cos(\gamma)$
a zero e o controlador de velocidade angular busca enviar um sinal $\omega$
para  tender o valor de $\gamma$ a zero. Perceba que quando o valor de $\gamma$
tende a zero, a velocidade linear vai tender a reduzir apenas
distância $l$. Como o modelo cinemático do robô espera receber como entrada
um vetor de velocidade linear em coordenadas cartesianas então é preciso
transformar de volta a velocidade $v$ 

\[
    \begin{bmatrix}
        v_x \\
        v_y \\
    \end{bmatrix}
    =
    \begin{bmatrix}
        v\cos(\theta) \\
        v\sin(\theta) \\
    \end{bmatrix}
\]
onde, $v_x$, $v_y$ são as velocidade lineares em coordenadas cartesianas.